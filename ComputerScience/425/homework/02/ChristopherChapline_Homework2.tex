\documentclass{article}%
\usepackage{amsmath}
\usepackage{amsfonts}
\usepackage{amssymb}
\usepackage{listings}
\usepackage{graphicx}
\usepackage{tikz}
\usepackage{hyperref}%
\usepackage[a4paper,includeheadfoot,margin=0.5in]{geometry}
\setcounter{MaxMatrixCols}{30}
%TCIDATA{OutputFilter=late$x2$.dll}
%TCIDATA{Version=5.00.0.2552}
%TCIDATA{CSTFile=40 LaTeX article.cst}
%TCIDATA{Created=Thursday, August 21, 2008 14:03:59}
%TCIDATA{LastRevised=Wednesday, October 01, 2014 12:46:33}
%TCIDATA{<META NAME="GraphicsSave" CONTENT="32">}
%TCIDATA{<META NAME="SaveForMode" CONTENT="1">}
%TCIDATA{<META NAME="DocumentShell" CONTENT="Standard LaTeX\Blank - Standard LaTeX Article">}
%TCIDATA{Language=American English}
\newtheorem{theorem}{Theorem}
\newtheorem{acknowledgement}[theorem]{Acknowledgement}
\newtheorem{algorithm}[theorem]{Algorithm}
\newtheorem{axiom}[theorem]{Axiom}
\newtheorem{case}[theorem]{Case}
\newtheorem{claim}[theorem]{Claim}
\newtheorem{conclusion}[theorem]{Conclusion}
\newtheorem{condition}[theorem]{Condition}
\newtheorem{conjecture}[theorem]{Conjecture}
\newtheorem{corollary}[theorem]{Corollary}
\newtheorem{criterion}[theorem]{Criterion}
\newtheorem{definition}[theorem]{Definition}
\newtheorem{example}[theorem]{Example}
\newtheorem{exercise}[theorem]{Exercise}
\newtheorem{lemma}[theorem]{Lemma}
\newtheorem{notation}[theorem]{Notation}
\newtheorem{problem}[theorem]{Problem}
\newtheorem{proposition}[theorem]{Proposition}
\newtheorem{remark}[theorem]{Remark}
\newtheorem{solution}[theorem]{Solution}
\newtheorem{summary}[theorem]{Summary}
\newenvironment{proof}[1][Proof]{\noindent\textbf{#1.} }{\ \rule{0.5em}{0.5em}}

\usepackage{fancyhdr}
\setlength\headheight{26pt}
\pagestyle{fancy}
\lhead{{\footnotesize Homework 2}}
\rhead{{\footnotesize Christopher Chapline}}
\begin{document}

\section*{Chapter 2}
\subsection*{Problem 5}
110101111101011111\textbf{0}101011111\textbf{0}110


\subsection*{Problem 11}
If the two erroneous bits are not in the same column, then the parity byte at the end of the frame will
have an incorrect number based on the count of 1s in those columns. If they occur in the same column,
then the parity bits at the end of each byte will reveal a mismatched number of 1s in the bytes that the
errors occurred in.


\subsection*{Problem 43a}
$A$ will have a $63\%$ chance to win the race. If the backoff time for $A$ is $0 \times T$, then $A$ will win under
any circumstance except for when the backoff time for $B$ is also $0 \times T$. If the backoff time for $A$ is $1 \times T$,
then $A$ will win as long as the backoff time for $B$ is either $2 \times T$ or $3 \times T$.


\section*{Chapter 3}
\subsection*{Problem 13}
The following ports would not be included:
\begin{enumerate}
    \item $B2$ to D
    \item $B5$ to B
    \item $B6$ to I
\end{enumerate}


\subsection*{Problem 17}

\subsubsection*{Part a}
Bridges $B1$, $B2$, and $B3$ will learn where X is. Y's interface will see the packet.

\subsubsection*{Part b}
Bridges $B1$, $B2$, and $B3$ will learn where Z is. Y's interface will not see the packet.

\subsubsection*{Part c}
Bridge $B1$ will learn where Y is. Z's interface will not see the packet.

\subsubsection*{Part d}
Bridge $B2$ will learn where W is. Z's interface will not see the packet.


\section*{Chapter 5}
\subsection*{Problem 9}
\subsubsection*{Part a}
The \textbf{AdvertisedWindow} field should be 25 bits. The \textbf{SequenceNumber} should be 32 bits to
prevent wrap around inside of the 30 second segment lifetime.

\subsubsection*{Part b}
The network speed would depend upon the hardware that you're network is built upon and would be relatively sure. The RTT might
fluctuate if there are issues with devices in the network (failures or congestion), so that is less certain. The segment lifetime
would be a design choice for your protocol and would thus be certain.


\section*{Chapter 6}
\subsection*{Problem 16}


\end{document}

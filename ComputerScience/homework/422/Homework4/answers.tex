\documentclass{article}%
\usepackage{amsmath}
\usepackage{amsfonts}
\usepackage{amssymb}
\usepackage{graphicx}
\usepackage{listings}
\usepackage{tikz}
\usepackage{hyperref}%
\setcounter{MaxMatrixCols}{30}
%TCIDATA{OutputFilter=late$x2$.dll}
%TCIDATA{Version=5.00.0.2552}
%TCIDATA{CSTFile=40 LaTeX article.cst}
%TCIDATA{Created=Thursday, August 21, 2008 14:03:59}
%TCIDATA{LastRevised=Wednesday, October 01, 2014 12:46:33}
%TCIDATA{<META NAME="GraphicsSave" CONTENT="32">}
%TCIDATA{<META NAME="SaveForMode" CONTENT="1">}
%TCIDATA{<META NAME="DocumentShell" CONTENT="Standard LaTeX\Blank - Standard LaTeX Article">}
%TCIDATA{Language=American English}
\newtheorem{theorem}{Theorem}
\newtheorem{acknowledgement}[theorem]{Acknowledgement}
\newtheorem{algorithm}[theorem]{Algorithm}
\newtheorem{axiom}[theorem]{Axiom}
\newtheorem{case}[theorem]{Case}
\newtheorem{claim}[theorem]{Claim}
\newtheorem{conclusion}[theorem]{Conclusion}
\newtheorem{condition}[theorem]{Condition}
\newtheorem{conjecture}[theorem]{Conjecture}
\newtheorem{corollary}[theorem]{Corollary}
\newtheorem{criterion}[theorem]{Criterion}
\newtheorem{definition}[theorem]{Definition}
\newtheorem{example}[theorem]{Example}
\newtheorem{exercise}[theorem]{Exercise}
\newtheorem{lemma}[theorem]{Lemma}
\newtheorem{notation}[theorem]{Notation}
\newtheorem{problem}[theorem]{Problem}
\newtheorem{proposition}[theorem]{Proposition}
\newtheorem{remark}[theorem]{Remark}
\newtheorem{solution}[theorem]{Solution}
\newtheorem{summary}[theorem]{Summary}
\newenvironment{proof}[1][Proof]{\noindent\textbf{#1.} }{\ \rule{0.5em}{0.5em}}

\usepackage{fancyhdr}
\setlength\headheight{26pt}
\pagestyle{fancy}
\lhead{{\footnotesize CSc 422 - Homework 4 Problems}}
\rhead{{\footnotesize Christopher Chapline}}
\begin{document}

\section*{Problem 1}

\subsection*{Part a}

\lstset{language=Python}
\begin{lstlisting}[frame=single]
chan in(int)
chan out1(int)
chan out2(int)

process Partition:
    int v;
    receive in(v)
    while( !empty(in) ):
        int next;
        receive in(next)

        if next <= v:
            send out1(next)
        else:
            send out2(next)
\end{lstlisting}\\
\\[0.1in]
We can make the following conclusions about the values present in the values entering and exiting the process:\\
\\
\textbf{out1}: $i \leq v, \forall i \in \textbf{out1}$
\hfill
\textbf{out2}: $i > v, \forall i \in \textbf{out2}$\\
\\
\textbf{in}: $v \in \textbf{in } \wedge \textbf{in} = \textbf{out1} \cup \textbf{out2}$

\subsection*{Part b}

\section*{Problem 2}
\section*{Problem 3}

\lstset{language=Python}
\begin{lstlisting}[frame=single]
chan fromA(A, B, int, int, chan response(bool))

process Server {
    while( (not empty(as)) and (not empty(bs)) ) {
        A a;
        B b;
        int numMetByA,
        int numMetByB,
        chan response;

        receive toServer(a, b, numMet, response)

        if(numMet
    }
}

\end{lstlisting}

\end{document}

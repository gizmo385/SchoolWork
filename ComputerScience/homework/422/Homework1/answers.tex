\documentclass{article}%
\usepackage{amsmath}
\usepackage{amsfonts}
\usepackage{amssymb}
\usepackage{graphicx}
\usepackage{tikz}
\usepackage{hyperref}%
\setcounter{MaxMatrixCols}{30}
%TCIDATA{OutputFilter=latex2.dll}
%TCIDATA{Version=5.00.0.2552}
%TCIDATA{CSTFile=40 LaTeX article.cst}
%TCIDATA{Created=Thursday, August 21, 2008 14:03:59}
%TCIDATA{LastRevised=Wednesday, October 01, 2014 12:46:33}
%TCIDATA{<META NAME="GraphicsSave" CONTENT="32">}
%TCIDATA{<META NAME="SaveForMode" CONTENT="1">}
%TCIDATA{<META NAME="DocumentShell" CONTENT="Standard LaTeX\Blank - Standard LaTeX Article">}
%TCIDATA{Language=American English}
\newtheorem{theorem}{Theorem}
\newtheorem{acknowledgement}[theorem]{Acknowledgement}
\newtheorem{algorithm}[theorem]{Algorithm}
\newtheorem{axiom}[theorem]{Axiom}
\newtheorem{case}[theorem]{Case}
\newtheorem{claim}[theorem]{Claim}
\newtheorem{conclusion}[theorem]{Conclusion}
\newtheorem{condition}[theorem]{Condition}
\newtheorem{conjecture}[theorem]{Conjecture}
\newtheorem{corollary}[theorem]{Corollary}
\newtheorem{criterion}[theorem]{Criterion}
\newtheorem{definition}[theorem]{Definition}
\newtheorem{example}[theorem]{Example}
\newtheorem{exercise}[theorem]{Exercise}
\newtheorem{lemma}[theorem]{Lemma}
\newtheorem{notation}[theorem]{Notation}
\newtheorem{problem}[theorem]{Problem}
\newtheorem{proposition}[theorem]{Proposition}
\newtheorem{remark}[theorem]{Remark}
\newtheorem{solution}[theorem]{Solution}
\newtheorem{summary}[theorem]{Summary}
\newenvironment{proof}[1][Proof]{\noindent\textbf{#1.} }{\ \rule{0.5em}{0.5em}}
\begin{document}

\title{CS 352 - Homework 1}
\author{Christopher Chapline}
\maketitle

\section{Problem 1}

\subsection{Part a}
\subsection{Part b}

\section{Problem 2}

\subsection{Part a}

$\forall i : 1 \leq i \leq m, \neg \exists j : 1 \leq j \leq n : a[i] \geq b[j]$

\subsection{Part c}

$\neg((\exists i : 1 \leq i \leq m : a[i] = 0) \wedge (\exists j : 1 \leq j \leq m : b[j] = 0))$

\subsection{Part e}

$\forall i : 1 \leq i \leq m, \exists j : 1 \leq j \leq n : a[i] = b[j]$

\section{Problem 3}

\subsection{Part a}

While it is possible that it could stop, it is highly unlikely that it will. For it to stop, the scheduler must stop the process executing the while
loop immediately when $x=10$ and it must start the process that checks for $x==0$ before it restarts the while loop process. While possible, this is
improbable

\subsection{Part b}

In the case that the scheduling is strongly fair,

\subsection{Part c.a}
\subsection{Part c.b}

\end{document}

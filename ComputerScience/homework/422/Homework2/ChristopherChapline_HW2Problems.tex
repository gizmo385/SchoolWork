\documentclass{article}%
\usepackage{amsmath}
\usepackage{amsfonts}
\usepackage{amssymb}
\usepackage{graphicx}
\usepackage{tikz}
\usepackage{hyperref}%
\setcounter{MaxMatrixCols}{30}
%TCIDATA{OutputFilter=late$x2$.dll}
%TCIDATA{Version=5.00.0.2552}
%TCIDATA{CSTFile=40 LaTeX article.cst}
%TCIDATA{Created=Thursday, August 21, 2008 14:03:59}
%TCIDATA{LastRevised=Wednesday, October 01, 2014 12:46:33}
%TCIDATA{<META NAME="GraphicsSave" CONTENT="32">}
%TCIDATA{<META NAME="SaveForMode" CONTENT="1">}
%TCIDATA{<META NAME="DocumentShell" CONTENT="Standard LaTeX\Blank - Standard LaTeX Article">}
%TCIDATA{Language=American English}
\newtheorem{theorem}{Theorem}
\newtheorem{acknowledgement}[theorem]{Acknowledgement}
\newtheorem{algorithm}[theorem]{Algorithm}
\newtheorem{axiom}[theorem]{Axiom}
\newtheorem{case}[theorem]{Case}
\newtheorem{claim}[theorem]{Claim}
\newtheorem{conclusion}[theorem]{Conclusion}
\newtheorem{condition}[theorem]{Condition}
\newtheorem{conjecture}[theorem]{Conjecture}
\newtheorem{corollary}[theorem]{Corollary}
\newtheorem{criterion}[theorem]{Criterion}
\newtheorem{definition}[theorem]{Definition}
\newtheorem{example}[theorem]{Example}
\newtheorem{exercise}[theorem]{Exercise}
\newtheorem{lemma}[theorem]{Lemma}
\newtheorem{notation}[theorem]{Notation}
\newtheorem{problem}[theorem]{Problem}
\newtheorem{subproblem}[theorem]{Problem}
\newtheorem{proposition}[theorem]{Proposition}
\newtheorem{remark}[theorem]{Remark}
\newtheorem{solution}[theorem]{Solution}
\newtheorem{summary}[theorem]{Summary}
\newenvironment{proof}[1][Proof]{\noindent\textbf{#1.} }{\ \rule{0.5em}{0.5em}}

\usepackage{fancyhdr}
\setlength\headheight{26pt}
\pagestyle{fancy}
\lhead{{\footnotesize CS 422 - Homework 2}}
\rhead{{\footnotesize Christopher Chapline}}
\begin{document}

\section*{Problem 1}


\section*{Problem 2}

\subsection*{Part (a)}
Yes because only one process will get a $P(d)$

No because $S$ never executes

\subsection*{Part (b)}
This implementation of $await$ statements doesn't avoid deadlock. Because there is no
call to $V(d)$, every process will block upon reaching the $P(d)$.

\subsection*{Part (c)}
This does not guarentee that $B$ is true when $S$ is executed. This guarentees that $B$ \underline{has been}
\textbf{true} before $S$ executes. Consider the possibility that after exiting the \textbf{while} loop, the
process is interrupted and another process changes $B$. In this case, when the first process resumes, it is
possible that $B$ is \textbf{false} when $S$ is executed.

\section*{Problem 3}

\subsection*{Part (a)}
The $e$ semaphor restricts access to the $nr$, $nw$, $dr$, and $dw$ counters. The $r$ and $w$ semaphors represent
the restrictions placed on when a reader or writer can enter the database. A reader will be allowed to enter their
critical section when there are no writers in the database. Conversely, a writer will be able to enter the database
when there are no readers or writers in the database. When a writer is waiting to enter a database, it will increment
a counter tracking the number of writers that are waiting. Similarly, when a reader is waiting to enter a database, it
will increment a counter to keep track of the number of readers waiting.

\subsection*{Part (b)}
This is a writers preference solution. In the Writers process, it will prefer to awake a delayed writer rather than a
delayed reader.

\subsection*{Part (c)}
One of the most immediate differences is that the solution in figure 4.13 is a readers preference.\\
\\
In the best case, the reader process in figure 4.21 will make 3 $P$ calls and 2 $V$ calls per iteration of the
\textbf{while} loop. In the worst case, the reader process will make 3 $P$ calls and 3 $V$ calls per iteration of the
\textbf{while} loop.\\
\\
For the writer process in figure 4.21, 3 $P$ calls and 2 $V$ calls will be made per iteration in the best case. In the
worst case, 3 $P$ calls and 4 $V$ calls will be made.\\
\\
For the reader process in figure 4.13, 2 $P$ calls and 2 $V$ calls will be made for each iteration of the loop in the
best case. In the worst case, 3 $P$ calls will be made and 3 $V$ calls will be made per loop iteration.\\
\\
For the writer process in figure 4.13, 2 $P$ calls and 2 $V$ calls will be made per iteration in the best case. In the
worst case, 3 $P$ calls and 3 $V$ calls will be made.\\




\end{document}

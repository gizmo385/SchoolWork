\documentclass{article}%
\usepackage{amsmath}
\usepackage{amsfonts}
\usepackage{amssymb}
\usepackage{graphicx}
\usepackage{tikz}
\usepackage{hyperref}%
\setcounter{MaxMatrixCols}{30}
%TCIDATA{OutputFilter=late$x2$.dll}
%TCIDATA{Version=5.00.0.2552}
%TCIDATA{CSTFile=40 LaTeX article.cst}
%TCIDATA{Created=Thursday, August 21, 2008 14:03:59}
%TCIDATA{LastRevised=Wednesday, October 01, 2014 12:46:33}
%TCIDATA{<META NAME="GraphicsSave" CONTENT="32">}
%TCIDATA{<META NAME="SaveForMode" CONTENT="1">}
%TCIDATA{<META NAME="DocumentShell" CONTENT="Standard LaTeX\Blank - Standard LaTeX Article">}
%TCIDATA{Language=American English}
\newtheorem{theorem}{Theorem}
\newtheorem{acknowledgement}[theorem]{Acknowledgement}
\newtheorem{algorithm}[theorem]{Algorithm}
\newtheorem{axiom}[theorem]{Axiom}
\newtheorem{case}[theorem]{Case}
\newtheorem{claim}[theorem]{Claim}
\newtheorem{conclusion}[theorem]{Conclusion}
\newtheorem{condition}[theorem]{Condition}
\newtheorem{conjecture}[theorem]{Conjecture}
\newtheorem{corollary}[theorem]{Corollary}
\newtheorem{criterion}[theorem]{Criterion}
\newtheorem{definition}[theorem]{Definition}
\newtheorem{example}[theorem]{Example}
\newtheorem{exercise}[theorem]{Exercise}
\newtheorem{lemma}[theorem]{Lemma}
\newtheorem{notation}[theorem]{Notation}
\newtheorem{problem}[theorem]{Problem}
\newtheorem{proposition}[theorem]{Proposition}
\newtheorem{remark}[theorem]{Remark}
\newtheorem{solution}[theorem]{Solution}
\newtheorem{summary}[theorem]{Summary}
\newenvironment{proof}[1][Proof]{\noindent\textbf{#1.} }{\ \rule{0.5em}{0.5em}}

\usepackage{fancyhdr}
\setlength\headheight{26pt}
\pagestyle{fancy}
\lhead{{\footnotesize CSc 473 - Homework 9}}
\rhead{{\footnotesize Christopher Chapline}}
\begin{document}

\section*{Problem 9.1}

\subsection*{Part a}

\subsection*{Part b}


\section*{Problem 9.2}

\subsection*{Part a}

\begin{tabular}{| x | y | z | e |}
    \hline
    \textbf{X} & \textbf{Y} & \textbf{Z} & \textbf{(x+ y)(–(x+z)) + (–x+ –z)} \\
    \hline
    T & T & T & \\
    \hline
    T & T & F & \\
    \hline
    T & F & T & \\
    \hline
    T & F & F & \\
    \hline
    F & T & T & \\
    \hline
    F & T & F & \\
    \hline
    F & F & T & \\
    \hline
    F & F & F & \\
    \hline
\end{tabular}

\subsection*{Part b}

\begin{tabular}{| x | y | z | e |}
    \hline
    \textbf{X} & \textbf{Y} & \textbf{Z} & \textbf{(x+ y) + (–(x+z) + (–x+ –z))} \\
    \hline
    T & T & T & \\
    \hline
    T & T & F & \\
    \hline
    T & F & T & \\
    \hline
    T & F & F & \\
    \hline
    F & T & T & \\
    \hline
    F & T & F & \\
    \hline
    F & F & T & \\
    \hline
    F & F & F & \\
    \hline
\end{tabular}

\subsection*{Part c}

The approach of trying every possible combination \textbf{true}/\textbf{false} value in the expression does generalize to an unbounded
number of variables. This is an incredibly inefficient algorithm that has a running time of $O(2^n)}$ where $n$ is the number of varaibles
in the expression.

\end{document}

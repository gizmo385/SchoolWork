\documentclass{article}%
\usepackage{amsmath}
\usepackage{amsfonts}
\usepackage{amssymb}
\usepackage{graphicx}
\usepackage{tikz}
\usepackage{hyperref}%
\setcounter{MaxMatrixCols}{30}
%TCIDATA{OutputFilter=late$x2$.dll}
%TCIDATA{Version=5.00.0.2552}
%TCIDATA{CSTFile=40 LaTeX article.cst}
%TCIDATA{Created=Thursday, August 21, 2008 14:03:59}
%TCIDATA{LastRevised=Wednesday, October 01, 2014 12:46:33}
%TCIDATA{<META NAME="GraphicsSave" CONTENT="32">}
%TCIDATA{<META NAME="SaveForMode" CONTENT="1">}
%TCIDATA{<META NAME="DocumentShell" CONTENT="Standard LaTeX\Blank - Standard LaTeX Article">}
%TCIDATA{Language=American English}
\newtheorem{theorem}{Theorem}
\newtheorem{acknowledgement}[theorem]{Acknowledgement}
\newtheorem{algorithm}[theorem]{Algorithm}
\newtheorem{axiom}[theorem]{Axiom}
\newtheorem{case}[theorem]{Case}
\newtheorem{claim}[theorem]{Claim}
\newtheorem{conclusion}[theorem]{Conclusion}
\newtheorem{condition}[theorem]{Condition}
\newtheorem{conjecture}[theorem]{Conjecture}
\newtheorem{corollary}[theorem]{Corollary}
\newtheorem{criterion}[theorem]{Criterion}
\newtheorem{definition}[theorem]{Definition}
\newtheorem{example}[theorem]{Example}
\newtheorem{exercise}[theorem]{Exercise}
\newtheorem{lemma}[theorem]{Lemma}
\newtheorem{notation}[theorem]{Notation}
\newtheorem{problem}[theorem]{Problem}
\newtheorem{proposition}[theorem]{Proposition}
\newtheorem{remark}[theorem]{Remark}
\newtheorem{solution}[theorem]{Solution}
\newtheorem{summary}[theorem]{Summary}
\newenvironment{proof}[1][Proof]{\noindent\textbf{#1.} }{\ \rule{0.5em}{0.5em}}

\usepackage{fancyhdr}
\setlength\headheight{26pt}
\pagestyle{fancy}
\lhead{{\footnotesize CSc 473 - Homework 8}}
\rhead{{\footnotesize Christopher Chapline}}
\begin{document}

\section*{Problem 1}

States: $\{s, r, f\}$ where $q$ is our initial state and $f$ is our final state\\
\\
Our Turing Machine is defined by the following instaneous descriptions:\\
\\
$\delta(q, 0) = (q, 0, R)$
\hfill
$\delta(q, 1) = (q, 1, R)$
\hfill
$\delta(q, B) = (r, B, L)$\\
\\
$\delta(r, 0) = (f, 1, R)$
\hfill
$\delta(r, 1) = (r, 0, L)$

\section*{Problem 2}

The method that we will use to perform binary addition is as follows: First of all, we will scan left-to-right copying the symbol from the left-hand operand
onto the second tape. Upon reaching a blank, we will switch states. Then, we will move the second tape to its start.Afterwards, we proceed through the second
operand, we will print onto the second tape the result of
adding the bits at the head of each tape.\\
\\
Here is an explanation of the states used in this Turing Machine:
\begin{itemize}
        \begin{item}
            $q_0$: This state will be the state that we use to copy the left-hand operand onto the second tape. This is also our start state.
        \end{item}

        \begin{item}
            $q_1$: This state will be the state that we use to move the first tape to the last digit on the tape
        \end{item}

        \begin{item}
            $q_2$: This is the state that will be used while scanning left-to-right on both the first and second tapes to perform the addition
        \end{item}

        \begin{item}
            $q_3$: This is the state that will be used when addition needs to have a carry
        \end{item}

        \begin{item}
            $f$: This is our final state and will indicate that the addition has been completed.
        \end{item}
\end{itemize}\\
\\
States: $\{q_0, q_1, q_2, f\}$\\
\\
We have two tapes. The first tape consists of the two operands, separated by a space. The second tape consists of all blanks, but will be the location where
we store the result of the addition.\\
\\
These are our transitions:\\
\\
% These transitions handle copying into the second tape
$\delta(q_0, 0, B) = (q_0, 0, 0, R, R)$
\hfill
$\delta(q_0, 1, B) = (q_0, 1, 1, R, R)$\\
\\
$\delta(q_0, B, B) = (q_1, B, B, S, L)$
\hfill
% These transitions handle moving to the end of the first tape
$\delta(q_1, 0, B) = (q_1, 0, B, R, S)$\\
\\
$\delta(q_1, 1, B) = (q_1, 1, B, R, S)$
\hfill
$\delta(q_1, B, B) = (q_2, B, B, R, S)$\\
\\
% These transitions handle addition
$\delta(q_2, 1, 1) = (q_2, 1, 0, L, L)$
\hfill
$\delta(q_2, 0, 1) = (q_2, 0, 1, L, L)$\\
\\
$\delta(q_2, 0, 0) = (q_2, 0, 0, L, L)$
\hfill
$\delta(q_2, 1, 1) = (q_3, 1, 1, L, L)$\\
\\
% These transitions handle addition with carries
$\delta(q_3, 0, 1) = (q_2, 0, 0, L, L)$
\hfill
$\delta(q_3, 1, 0) = (q_2, 1, 0, L, L)$\\
\\
$\delta(q_3, 0, 0) = (q_2, 0, 1, L, L)$
\hfill
$\delta(q_3, 1, 1) = (q_3, 1, 0, L, L)$\\
\\
% This transition handles transitioning into the final state
$\delta(q_2, B, B) = (f, B, B, S, S)$\\
\\

\section*{Problem 3}

We need the following states:
\begin{itemize}
        \begin{item}
            $s$: This is our start state
        \end{item}

        \begin{item}
            $q_0$: This state represents the situation where we're writing a 0 on the next cell
        \end{item}

        \begin{item}
            $q_1$: This state represents the situation where we're writing a 1 on the next cell
        \end{item}

        \begin{item}
            $q_{b0}$: This represents the situation where we're writing a B in the next cell and a zero in the cell after that
        \end{item}

        \begin{item}
            $q_{b1}$: This represents the situation where we're writing a B in the next cell and a one in the cell after that
        \end{item}

        \begin{item}
            $f$: This represents our final state, when we have nothing to write and are on a B
        \end{item}
\end{itemize}\\
\\
Here are our transitions:\\
\\
% Transitions out of the start state
$\delta(s, 0) = (q_0, 0, R)$
\hfill
$\delta(s, 1) = (q_1, 1, R)$
\hfill
$\delta(s, B) = (f, B, S)$\\
\\
% Transitions handling q0
$\delta(q_0, 0) = (q_{b0}, 0, R)$
\hfill
$\delta(q_0, 1) = (q_{b1}, 0, R)$
\hfill
$\delta(q_0, B) = (f, 0, R)$\\
\\
%Transitions handling q1
$\delta(q_1, 0) = (q_{b0}, 1, R)$
\hfill
$\delta(q_1, 1) = (q_{b1}, 1, R)$
\hfill
$\delta(q_1, B) = (f, 1, R)$\\
\\
% b0 transitions
$\delta(q_{b0}, 0) = (q_{b0}, B, R)$
\hfill
$\delta(q_{b0}, 1) = (q_{b1}, B, R)$
\hfill
$\delta(q_{b0}, B) = (q_{0}, B, R)$\\
\\
% b1 transitions
$\delta(q_{b1}, 0) = (q_{b0}, B, R)$
\hfill
$\delta(q_{b1}, 1) = (q_{b1}, B, R)$
\hfill
$\delta(q_{b1}, B) = (q_{1}, B, R)$\\
\\


\section*{Problem 4}

Each string will require us to move the length of the string plus the number of B's that will need to be inserted into the string, which is $m-1$. Thus, we could
define the function like so:

$f(m) = m + (m - 1)$


\end{document}

\documentclass{article}%
\usepackage{amsmath}
\usepackage{amsfonts}
\usepackage{amssymb}
\usepackage{listings}
\usepackage{graphicx}
\usepackage{tikz}
\usepackage{hyperref}%
\usepackage[a4paper,includeheadfoot,margin=3cm]{geometry}
\setcounter{MaxMatrixCols}{30}
%TCIDATA{OutputFilter=late$x2$.dll}
%TCIDATA{Version=5.00.0.2552}
%TCIDATA{CSTFile=40 LaTeX article.cst}
%TCIDATA{Created=Thursday, August 21, 2008 14:03:59}
%TCIDATA{LastRevised=Wednesday, October 01, 2014 12:46:33}
%TCIDATA{<META NAME="GraphicsSave" CONTENT="32">}
%TCIDATA{<META NAME="SaveForMode" CONTENT="1">}
%TCIDATA{<META NAME="DocumentShell" CONTENT="Standard LaTeX\Blank - Standard LaTeX Article">}
%TCIDATA{Language=American English}
\newtheorem{theorem}{Theorem}
\newtheorem{acknowledgement}[theorem]{Acknowledgement}
\newtheorem{algorithm}[theorem]{Algorithm}
\newtheorem{axiom}[theorem]{Axiom}
\newtheorem{case}[theorem]{Case}
\newtheorem{claim}[theorem]{Claim}
\newtheorem{conclusion}[theorem]{Conclusion}
\newtheorem{condition}[theorem]{Condition}
\newtheorem{conjecture}[theorem]{Conjecture}
\newtheorem{corollary}[theorem]{Corollary}
\newtheorem{criterion}[theorem]{Criterion}
\newtheorem{definition}[theorem]{Definition}
\newtheorem{example}[theorem]{Example}
\newtheorem{exercise}[theorem]{Exercise}
\newtheorem{lemma}[theorem]{Lemma}
\newtheorem{notation}[theorem]{Notation}
\newtheorem{problem}[theorem]{Problem}
\newtheorem{proposition}[theorem]{Proposition}
\newtheorem{remark}[theorem]{Remark}
\newtheorem{solution}[theorem]{Solution}
\newtheorem{summary}[theorem]{Summary}
\newenvironment{proof}[1][Proof]{\noindent\textbf{#1.} }{\ \rule{0.5em}{0.5em}}

\usepackage{fancyhdr}
\setlength\headheight{26pt}
\pagestyle{fancy}
\lhead{{\footnotesize CSc 483 - Assignment 2}}
\rhead{{\footnotesize Christopher Chapline}}
\begin{document}
\section*{Problem 1}
\subsubsection*{'Cos}
The word 'cos is a shortened form of the word because, the normalized form should be \textbf{because}.
\subsubsection*{Shi'ite}
The word Shi'ite should be normalized to \textbf{Shiite} based on the rationale that most people will not add the apostrophe while searching for it.
\subsubsection*{cont'd}
The word cont'd should be normalized to \textbf{continue}. The word is a shortened form of the word "continued", but we should normalize all forms of the word to present tense to catch as many results as possible during searching.
\subsubsection*{Hawai'i}
The word Hawai'i should be normalized to \textbf{Hawaii} under the same rationale that was used to normalize Shi'ite: most people will not type the apostrophe.
\subsubsection*{O'Rourke}
This should be normalized to \textbf{Orourke}. This follows from the same justification used for Hawai'i and Shi'ite: users are lazy.
\subsubsection*{ain't}
This word is a contraction of the phrase is not. Since that is the case, we should normalize the word into \textbf{is not}.
\section*{Problem 2}

The newly constructed \underline{York University} has recently opened for admission in \underline{New York} and will serve students across all 5 boroughs.

\section*{Problem 3}
The query \textit{"fools rush in"} returns documents 2, 4, and 7.\\
\\
The query \textit{"fools rush in" AND "angels fear to tread"} returns only document 4.
\section*{Problem 4}
\section*{Problem 5}
\begin{tabular}{| l | l | l | l | l | l | l |}
    \hline
    & & p & a & r & i & s \\ \hline
    & 0 & 1 & 2  & 3  & 4  & 5  \\ \hline
    % Row 1
    a & 1 &
    \begin{tabular}{l | l}
        1 & 2 \\ \hline 2 & 1
    \end{tabular}
    &
    \begin{tabular}{l | l}
        1 & 3 \\ \hline 2 & 1
    \end{tabular}
    &
    \begin{tabular}{l | l}
        3 & 4 \\ \hline 2 & 2
    \end{tabular}
    &
    \begin{tabular}{l | l}
        4 & 5 \\ \hline 3 & 3
    \end{tabular}
    &
    \begin{tabular}{l | l}
        5 & 6 \\ \hline 4 & 4 \\
    \end{tabular} \\ \hline

    % Row 2
    l & 2 &
    \begin{tabular}{l | l}
        2 & 2 \\ \hline 3 & 2
    \end{tabular}
    &
    \begin{tabular}{l | l}
        3 & 2 \\ \hline 3 & 2
    \end{tabular}
    &
    \begin{tabular}{l | l}
        2 & 3 \\ \hline 3 & 2
    \end{tabular}
    &
    \begin{tabular}{l | l}
        3 & 4 \\ \hline 3 & 3
    \end{tabular}
    &
    \begin{tabular}{l | l}
        4 & 5 \\ \hline 4 & 4 \\
    \end{tabular} \\ \hline

    % Row 3
    i & 3 &
    \begin{tabular}{l | l}
        3 & 3 \\ \hline 4 & 3
    \end{tabular}
    &
    \begin{tabular}{l | l}
        3 & 3 \\ \hline 4 & 3
    \end{tabular}
    &
    \begin{tabular}{l | l}
        3 & 3 \\ \hline 4 & 3
    \end{tabular}
    &
    \begin{tabular}{l | l}
        2 & 4 \\ \hline 4 & 2
    \end{tabular}
    &
    \begin{tabular}{l | l}
        4 & 5 \\ \hline 3 & 3 \\
    \end{tabular} \\ \hline

    % Row 4
    c & 4 &
    \begin{tabular}{l | l}
        4 & 4 \\ \hline 5 & 4
    \end{tabular}
    &
    \begin{tabular}{l | l}
        4 & 4 \\ \hline 5 & 4
    \end{tabular}
    &
    \begin{tabular}{l | l}
        4 & 4 \\ \hline 5 & 4
    \end{tabular}
    &
    \begin{tabular}{l | l}
        4 & 3 \\ \hline 5 & 3
    \end{tabular}
    &
    \begin{tabular}{l | l}
        3 & 4 \\ \hline 4 & 3 \\
    \end{tabular} \\ \hline

    % Row 5
    e & 5 &
    \begin{tabular}{l | l}
        5 & 5 \\ \hline 6 & 5
    \end{tabular}
    &
    \begin{tabular}{l | l}
        5 & 5 \\ \hline 6 & 5
    \end{tabular}
    &
    \begin{tabular}{l | l}
        5 & 5 \\ \hline 6 & 5
    \end{tabular}
    &
    \begin{tabular}{l | l}
        5 & 4 \\ \hline 5 & 4
    \end{tabular}
    &
    \begin{tabular}{l | l}
        4 & 4 \\ \hline 5 & \textbf{4} \\
    \end{tabular} \\ \hline
\end{tabular}\\
\\
As shown above, the Levenshtein distance between \textit{alice} and \textit{paris} is \textbf{4}.
\end{document}

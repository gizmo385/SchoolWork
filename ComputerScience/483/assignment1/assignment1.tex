\documentclass{article}%
\usepackage{amsmath}
\usepackage{amsfonts}
\usepackage{amssymb}
\usepackage{graphicx}
\usepackage{tikz}
\usepackage{hyperref}%
\setcounter{MaxMatrixCols}{30}
%TCIDATA{OutputFilter=late$x2$.dll}
%TCIDATA{Version=5.00.0.2552}
%TCIDATA{CSTFile=40 LaTeX article.cst}
%TCIDATA{Created=Thursday, August 21, 2008 14:03:59}
%TCIDATA{LastRevised=Wednesday, October 01, 2014 12:46:33}
%TCIDATA{<META NAME="GraphicsSave" CONTENT="32">}
%TCIDATA{<META NAME="SaveForMode" CONTENT="1">}
%TCIDATA{<META NAME="DocumentShell" CONTENT="Standard LaTeX\Blank - Standard LaTeX Article">}
%TCIDATA{Language=American English}
\newtheorem{theorem}{Theorem}
\newtheorem{acknowledgement}[theorem]{Acknowledgement}
\newtheorem{algorithm}[theorem]{Algorithm}
\newtheorem{axiom}[theorem]{Axiom}
\newtheorem{case}[theorem]{Case}
\newtheorem{claim}[theorem]{Claim}
\newtheorem{conclusion}[theorem]{Conclusion}
\newtheorem{condition}[theorem]{Condition}
\newtheorem{conjecture}[theorem]{Conjecture}
\newtheorem{corollary}[theorem]{Corollary}
\newtheorem{criterion}[theorem]{Criterion}
\newtheorem{definition}[theorem]{Definition}
\newtheorem{example}[theorem]{Example}
\newtheorem{exercise}[theorem]{Exercise}
\newtheorem{lemma}[theorem]{Lemma}
\newtheorem{notation}[theorem]{Notation}
\newtheorem{problem}[theorem]{Problem}
\newtheorem{proposition}[theorem]{Proposition}
\newtheorem{remark}[theorem]{Remark}
\newtheorem{solution}[theorem]{Solution}
\newtheorem{summary}[theorem]{Summary}
\newenvironment{proof}[1][Proof]{\noindent\textbf{#1.} }{\ \rule{0.5em}{0.5em}}

\usepackage{fancyhdr}
\setlength\headheight{26pt}
\pagestyle{fancy}
\lhead{{\footnotesize CSc 483 - Assignment 1}}
\rhead{{\footnotesize Christopher Chapline}}
\begin{document}

\section*{Problem 1}
\subsection*{Part 1}

\begin{tabular}{| a | b | c | d | e |}
\hline
& Doc 1 & Doc 2 & Doc 3 & Doc 4 \\ \hline
schizophrenia & 1 & 1 & 1 & 1 \\ \hline
drug & 1 & 1 & 0 & 0 \\ \hline
of & 0 & 0 & 1 & 0 \\ \hline
new & 0 & 1 & 1 & 1 \\ \hline
for & 1 & 0 & 1 & 1 \\ \hline
hopes & 0 & 0 & 0 & 1 \\ \hline
approach & 0 & 0 & 1 & 0 \\ \hline
patients & 0 & 0 & 0 & 1 \\ \hline
treatment & 0 & 0 & 1 & 0 \\ \hline
breakthrough & 1 & 0 & 0 & 0 \\ \hline
\end{tabular}

\subsection*{Part 2}
\begin{tabular}{| a | b | c | d | e |}
\hline
schizophrenia & 1 & 2 & 3 & 4 \\ \hline
drug & 1 & 2\\ \hline
of & 3 \\ \hline
new & 2 & 3 & 4 \\ \hline
for & 1 & 3 & 4 \\ \hline
hopes & 4 \\ \hline
approach & 3 \\ \hline
patients & 4 \\ \hline
treatment & 3 \\ \hline
breakthrough & 1 \\ \hline
\end{tabular}

\subsection*{Part 3}

\subsubsection*{Part a}

This query will yield the following documents: Doc 1, Doc 2

\subsubsection*{Part b}

Let's break this query down. The subquery "(drug OR approach)" will yield documents 1, 2, and 3. The subquery "for" will yield documents 1, 2, and 4. Applying the "AND NOT" operator to these results will leave us with only document 4.\\
\\
Thus, the ultimate result is document 4.

\section*{Problem 2}
\subsection*{Part 1}
\subsection*{Part 2}
\section*{Problem 3}
\section*{Problem 4}

\end{document}

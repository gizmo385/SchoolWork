\documentclass{article}%
\usepackage{amsmath}
\usepackage{amsfonts}
\usepackage{amssymb}
\usepackage{graphicx}
\usepackage{tikz}
\usepackage{hyperref}%
\usepackage[a4paper,includeheadfoot,margin=0.5in]{geometry}
\setcounter{MaxMatrixCols}{30}
%TCIDATA{OutputFilter=late$x2$.dll}
%TCIDATA{Version=5.00.0.2552}
%TCIDATA{CSTFile=40 LaTeX article.cst}
%TCIDATA{Created=Thursday, August 21, 2008 14:03:59}
%TCIDATA{LastRevised=Wednesday, October 01, 2014 12:46:33}
%TCIDATA{<META NAME="GraphicsSave" CONTENT="32">}
%TCIDATA{<META NAME="SaveForMode" CONTENT="1">}
%TCIDATA{<META NAME="DocumentShell" CONTENT="Standard LaTeX\Blank - Standard LaTeX Article">}
%TCIDATA{Language=American English}
\newtheorem{theorem}{Theorem}
\newtheorem{acknowledgement}[theorem]{Acknowledgement}
\newtheorem{algorithm}[theorem]{Algorithm}
\newtheorem{axiom}[theorem]{Axiom}
\newtheorem{case}[theorem]{Case}
\newtheorem{claim}[theorem]{Claim}
\newtheorem{conclusion}[theorem]{Conclusion}
\newtheorem{condition}[theorem]{Condition}
\newtheorem{conjecture}[theorem]{Conjecture}
\newtheorem{corollary}[theorem]{Corollary}
\newtheorem{criterion}[theorem]{Criterion}
\newtheorem{definition}[theorem]{Definition}
\newtheorem{example}[theorem]{Example}
\newtheorem{exercise}[theorem]{Exercise}
\newtheorem{lemma}[theorem]{Lemma}
\newtheorem{notation}[theorem]{Notation}
\newtheorem{problem}[theorem]{Problem}
\newtheorem{proposition}[theorem]{Proposition}
\newtheorem{remark}[theorem]{Remark}
\newtheorem{solution}[theorem]{Solution}
\newtheorem{summary}[theorem]{Summary}
\newenvironment{proof}[1][Proof]{\noindent\textbf{#1.} }{\ \rule{0.5em}{0.5em}}

\usepackage{fancyhdr}
\setlength\headheight{26pt}
\pagestyle{fancy}
\lhead{{\footnotesize Assignment 4}}
\rhead{{\footnotesize Christopher Chapline}}
\begin{document}

\section*{Problem 1}
We must first create contingency tables for each of the terms in our query:\\
\\
Contingency table for "obama": \\
\begin{tabular}{| l | l | l | l |}
    \hline
    Documents                   &  Relevant     & Nonrelevant   & Total         \\ \hline
    "obama" Present $x_t = 1$   &  $s = 1.5$    & 2.5           & $df_t = 4$    \\ \hline
    "obama" Absent $x_t = 0$    &  0.5          & 0.5           & 1           \\ \hline
    Total                       &  $S = 2$    & 3               & $N = 5$       \\ \hline
\end{tabular}\\
\\
$p_t = \frac{s}{S} = \frac{\frac{3}{2}}{2} = \frac{3}{4}$ \hfill
$u_t = \frac{df_t - s}{N - S} = \frac{4 - \frac{3}{2}}{5 - 2} = \frac{5}{6}$
\vspace{2mm}\\
$c_t = log \frac{p_t}{1- p_t} - log \frac{u_t}{1 - u_t}
= log \left(\frac{\frac{3}{4}}{1 - \frac{3}{4}}\right) - log \left(\frac{\frac{5}{6}}{1 - \frac{5}{6}}\right)
= log(3) - log(5) \approx -0.222$
\vspace{5mm}\\
Contingency table for "health": \\
\begin{tabular}{| l | l | l | l |}
    \hline
    Documents                   &  Relevant     & Nonrelevant   & Total         \\ \hline
    "health" Present $x_t = 1$  &  $s = 1.5$    & 1.5           & $df_t = 3$    \\ \hline
    "health" Absent $x_t = 0$   &  0.5          & 1.5           & 2             \\ \hline
    Total                       &  $S = 2$      & 3             & $N = 5$       \\ \hline
\end{tabular}\\
\\
$p_t = \frac{s}{S} = \frac{\frac{3}{2}}{2} = \frac{3}{4}$ \hfill
$u_t = \frac{df_t - s}{N - S} = \frac{3 - \frac{3}{2}}{5 - 2} = \frac{1}{2}$
\vspace{2mm}\\
$c_t = log \frac{p_t}{1- p_t} - log \frac{u_t}{1 - u_t}
= log \left(\frac{\frac{3}{4}}{1 - \frac{3}{4}}\right) - log \left(\frac{\frac{1}{2}}{1 - \frac{1}{2}}\right)
= log(3) - log(1) \approx 0.477$
\vspace{5mm}\\
Contingency table for "plan": \\
\begin{tabular}{| l | l | l | l |}
    \hline
    Documents                   &  Relevant     & Nonrelevant   & Total         \\ \hline
    "plan" Present $x_t = 1$    &  $s = 1.5$    & 1.5           & $df_t = 3$    \\ \hline
    "plan" Absent $x_t = 0$     &  0.5          & 1.5           & 2             \\ \hline
    Total                       &  $S = 2$      & 3             & $N = 5$       \\ \hline
\end{tabular}\\
$p_t = \frac{s}{S} = \frac{\frac{3}{2}}{2} = \frac{3}{4}$ \hfill
$u_t = \frac{df_t - s}{N - S} = \frac{3 - \frac{3}{2}}{5 - 2} = \frac{1}{2}$
\vspace{2mm}\\
$c_t = log \frac{p_t}{1- p_t} - log \frac{u_t}{1 - u_t}
= log \frac{\frac{3}{4}}{1- \frac{3}{4}} - log \frac{\frac{1}{2}}{1 - \frac{1}{2}}
= log(3) - log(1) \approx 0.477$
\vspace{5mm}\\
Now that we have calculated the $c_t$ value for each of the terms in our query, we can now calculate the RSV value for each of the documents:\\
\\
Doc1: $\sum_{x_t=q_t=1}c_t = -0.222 + 0.477 = 0.255$\\
\\
Doc2: $\sum_{x_t=q_t=1}c_t = -0.222 + 0.477 = 0.255$\\
\\
Doc3: $\sum_{x_t=q_t=1}c_t = -0.222 + 0.477 + 0.477 = 0.732$\\
\section*{Problem 2}
Consider the following set of words and their English classification:\\
\begin{tabular}{| l | l | l | l |}
    \hline
    event   & word  & English?  & probability \\ \hline
    1       & ozb   & no        & $\frac{4}{9}$ \\ \hline
    2       & uzu   & no        & $\frac{4}{9}$ \\ \hline
    3       & zoo   & yes       & $\frac{1}{18}$ \\ \hline
    4       & bun   & yes       & $\frac{1}{18}$ \\ \hline
\end{tabular}
\subsection*{Part 1}
The priors are: $P(\text{English}) = \frac{1}{9}$ and $P(\neg{\text{English}}) = \frac{8}{9}$.
Calculating probabilities for individual letters leads us to the following table:\\
\\
\begin{tabular}{| l | l | l |}
    \hline
        & $P(t | \text{English})$   & $P(t | \neg{\text{English}})$ \\ \hline
    b   & $0.17$             & $0.17$ \\ \hline
    n   & $0.17$             & $0.01$ \\ \hline
    o   & $0.33$             & $0.17$ \\ \hline
    u   & $0.17$             & $0.33$ \\ \hline
    z   & $0.17$             & $0.33$ \\ \hline
\end{tabular}
\subsection*{Part 2}
Now that we have our priors and probabilities, we can determine whether or not "zoo" would be classified as English. There are
two probabilities that we must consider:

$P(\text{English} | \text{zoo}) \propto
P(\text{z} | \text{English}) \cdot P(\text{o} | \text{English})^2 \cdot P(\text{English})
\approx 0.17 \cdot 0.33^2 \cdot 0.11
= 0.002
$

$P(\neg\text{English} | \text{zoo}) \propto
P(\text{z} | \neg\text{English}) \cdot P(\text{o} | \neg\text{English})^2 \cdot P(\neg\text{English})
\approx 0.33 \cdot 0.17^2 \cdot 0.89
= 0.008$\\
\\
Since the probability $P(\neg\text{English} | \text{zoo})$ is higher than that of $P(\text{English} | \text{zoo})$, we can classify
zoo as being non-English.
\section*{Problem 3}
Consider the following series of documents:\\
\\
\begin{tabular}{l | l}
    docID & Document Text \\ \hline
    1 & click go the shears boys click click click \\ \hline
    2 & click click \\ \hline
    3 & metal here \\ \hline
    4 & metal shears click here \\ \hline
\end{tabular}\\
\vspace{5mm}\\
Computing the probability for a query based upon a document will be done using the following formula:

$P(q | d) = \prod \left( \lambda P(t | M_d) + (1 - \lambda) P(t | M_c)\right)$\\
\\
In this formula, $\lambda = 0.5$, $M_d$ is the language model for the document and $M_c$ is the language model for the entire collection.
To compute these, we need to generate language models across our collection for the terms in each query. Additionally, for each query, we'll
need to create language models for each document in the collection.\\
\\
First we will compute the $M_c$ values for each of the terms in the three queries.\\
\begin{tabular}{| l | l |}
    \hline
    query term  & $M_c$ \\ \hline
    click       & $P(\text{click} | c) =  0.44$ \\ \hline
    shears      & $P(\text{shears} | c) =  0.13$ \\ \hline
\end{tabular}\\
\\
Now we will compute the $M_d$ values for each document for each of the terms in the three queries:\\
\begin{tabular}{| l | l | l | l | l | l |}
    \hline
    query term  & $M_{Doc1}$    & $M_{Doc2}$    & $M_{Doc3}$    & $M_{Doc4}$ \\ \hline
    click       & 0.5           & 1             & 0.00          & 0.25 \\ \hline
    shears      & 0.13          & 0.00          & 0.00          & 0.25 \\ \hline
\end{tabular}\\
\\
Here are the probabilities for three queries in this language model broken into two tables:\\
\\
\begin{tabular}{| l | l | l | l |}
    \hline
    Query           & $\lambda M_c$ & Doc1                                  & Doc2 \\ \hline
    click           &  $0.22$       & $0.22 + 0.50(0.50) = 0.47$            & $0.22 + 0.50(1.00) = 0.72$ \\ \hline
    shears          &  $0.07$       & $0.07 + 0.50(0.13) = 0.14$            & $0.07 + 0.50(0.00) = 0.07$ \\ \hline
    click shears    &  $0.03$       & $0.03 + 0.50(0.50 \cdot 0.13) = .06$  & $0.03 + 0.50(1.00 * 0.00) = 0.03$ \\ \hline
\end{tabular}\\
\vspace{2mm}
\\
\begin{tabular}{| l | l | l | l |}
    \hline
    Query           & $\lambda M_c$ & Doc3                                  & Doc4 \\ \hline
    click           &  $0.22$       & $0.22 + 0.50(0.00) = 0.22$            & $0.22 + 0.50(0.25) = 0.35$ \\ \hline
    shears          &  $0.07$       & $0.07 + 0.50(0.00) = 0.07$            & $0.07 + 0.50(0.25) = 0.20$ \\ \hline
    click shears    &  $0.03$       & $0.03 + 0.50(0.00 \cdot 0.00) = .03$  & $0.03 + 0.50(0.25 * 0.025) = 0.06$ \\ \hline
\end{tabular}\\
\\
Using these values as $P(q | d)$ values, we can rank the documents for each query:
\begin{itemize}
    \item "click" - Doc2, Doc1, Doc4, Doc3
    \item "shears" - Doc4, Doc1, Doc2, Doc3
    \item "click shears" - Doc1, Doc4, Doc2, Doc3
\end{itemize}
\section*{Problem 4}
\subsection*{Part 1}
\subsection*{Part 2}
\subsection*{Part 3}

\end{document}

\documentclass{article}%
\usepackage{amsmath}
\usepackage{amsfonts}
\usepackage{amssymb}
\usepackage{graphicx}
\usepackage{tikz}
\usepackage{hyperref}%
\setcounter{MaxMatrixCols}{30}
%TCIDATA{OutputFilter=late$x2$.dll}
%TCIDATA{Version=5.00.0.2552}
%TCIDATA{CSTFile=40 LaTeX article.cst}
%TCIDATA{Created=Thursday, August 21, 2008 14:03:59}
%TCIDATA{LastRevised=Wednesday, October 01, 2014 12:46:33}
%TCIDATA{<META NAME="GraphicsSave" CONTENT="32">}
%TCIDATA{<META NAME="SaveForMode" CONTENT="1">}
%TCIDATA{<META NAME="DocumentShell" CONTENT="Standard LaTeX\Blank - Standard LaTeX Article">}
%TCIDATA{Language=American English}
\newtheorem{theorem}{Theorem}
\newtheorem{acknowledgement}[theorem]{Acknowledgement}
\newtheorem{algorithm}[theorem]{Algorithm}
\newtheorem{axiom}[theorem]{Axiom}
\newtheorem{case}[theorem]{Case}
\newtheorem{claim}[theorem]{Claim}
\newtheorem{conclusion}[theorem]{Conclusion}
\newtheorem{condition}[theorem]{Condition}
\newtheorem{conjecture}[theorem]{Conjecture}
\newtheorem{corollary}[theorem]{Corollary}
\newtheorem{criterion}[theorem]{Criterion}
\newtheorem{definition}[theorem]{Definition}
\newtheorem{example}[theorem]{Example}
\newtheorem{exercise}[theorem]{Exercise}
\newtheorem{lemma}[theorem]{Lemma}
\newtheorem{notation}[theorem]{Notation}
\newtheorem{problem}[theorem]{Problem}
\newtheorem{proposition}[theorem]{Proposition}
\newtheorem{remark}[theorem]{Remark}
\newtheorem{solution}[theorem]{Solution}
\newtheorem{summary}[theorem]{Summary}
\newenvironment{proof}[1][Proof]{\noindent\textbf{#1.} }{\ \rule{0.5em}{0.5em}}

\usepackage{fancyhdr}
\setlength\headheight{26pt}
\pagestyle{fancy}
\lhead{{\footnotesize CSc 473 - Home &work 7}}
\rhead{{\footnotesize Christopher Chapline}}
\begin{document}

\section*{Problem 7.1}

\subsection*{Part b}

$(q, 0011, Z_0) \vdash (q, 011, XZ_0) \vdash (q, 11, XXZ_0) \vdash (q, 1, XXZ_0) \vdash (q, \epsilon, XXZ_0) \vdash (p, \epsilon, XZ_0) \vdash (p, \epsilon, Z_0) \vdash (p, \epsilon, \epsilon)$

\subsection*{Part c}

$(q, 010, Z_0) \vdash (q, 10, XZ_0) \vdash (q, 0, XXZ_0) \vdash (q, \epsilon, XXXZ_0) \vdash (p, \epsilon, XXZ_0) \vdash (p, \epsilon, XXZ_0) \vdash (p, \epsilon, XZ_0) \vdash (p, \epsilon, Z_0) \vdash (p, \epsilon, \epsilon)$

\section*{Problem 7.2}
\subsection*{Part a}

Here is the definition for the transition function, $\delta$:\\
\\
$\delta(q, \epsilon, E) = \{(q, TE')\}$\\
\\
$\delta(q, \epsilon, E') = \{(q, +TE'), (q, -TE')\}$\\
\\
$\delta(q, \epsilon, T) = \{(q, FT')\}$\\
\\
$\delta(q, \epsilon, T') = \{(q, *FT'), (q, /FT')\}$\\
\\
$\delta(q, \epsilon, T') = \{(q, /FT')\}$\\
\\
$\delta(q, \epsilon, \epsilon) = \{(q, \epsilon)\}$\\
\\
$\delta(q, +, +) = \{(q, \epsilon)\}$\\
\\
$\delta(q, -, -) = \{(q, \epsilon)\}$\\
\\
$\delta(q, a, a) = \{(q, \epsilon)\}$\\
\\
$\delta(q, b, b) = \{(q, \epsilon)\}$\\
\\
$\delta(q, c, c) = \{(q, \epsilon)\}$\\
\\

\subsection*{Part b}

$(a, a+b*c, Z_0) \vdash (q, +b*c, aZ_0) \vdash (q, +b*c, +TE'aZ_0) \vdash (q, b*c, TE'aZ_0) \vdash (q, b*c, FT'E'aZ_0) \vdash (q, b*c, bT'E'aZ_0) \vdash (q, *c, T'E'aZ_0) \vdash (q, *c, *FT'E'aZ_0)
\vdash (q, c, FT'E'aZ_0) \vdash (q, c, c\epsilon E'aZ_0) \vdash (q, \epsilon, \epsilon aZ_0) \vdash (q, \epsilon \epsilon)$

\section*{Problem 7.3}
\subsection*{Part a}

$X_{i,j}$ represents the variables that derive the string $a_i...a_j$.

\subsection*{Part b}

For cell 14, you need to consider cells 13 and 41, 24 and 23\\
\\
For cell 13, you need to consider cells 12 and 22, 11 and 23\\
\\
For cell 24, you need to consider 23 and 33, 22 and 34\\
\\


\subsection*{Part c}
\begin{tabular}{| l | l | l | l |}
    \hline
    $\{N, S\}$ & - & - & - \\
    \hline
    $\{N, S\}$ & $\{N\}$ & -  & - \\
    \hline
    $\{N\}$ & $\{N, P\}$ & $\{N\}$ & - \\
    \hline
    $\{N\}$ & $\{N, V\}$ & $\{N\}$ & $\{N\}$\\
    \hline
    Arizona & trains & astronomy & students \\
    \hline
\end{tabular}\\
\\
Because the top left cell contains $S$, the string is in the language.

\subsection*{Part d}

\begin{tabular}{| l | l | l | l |}
    \hline
    $\{N\}$ & & & \\
    \hline
    $\{N\}$ & $\{N\}$ & & \\
    \hline
    $\{N\}$ & $\{N, P\}$ & $\{N\}$ & \\
    \hline
    $\{N\}$ & $\{N, V\}$ & $\{N, V\}$ & $\{N\}$ \\
    \hline
    Arizona & trains & trains & students \\
    \hline
\end{tabular}\\
\\
Because the top left cell doesn't contain $S$, the string is not in the language.




\end{document}

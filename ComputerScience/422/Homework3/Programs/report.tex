\documentclass{article}%
\usepackage{amsmath}
\usepackage{amsfonts}
\usepackage{amssymb}
\usepackage{listings}
\usepackage{graphicx}
\usepackage{tikz}
\usepackage{hyperref}%
\setcounter{MaxMatrixCols}{30}
%TCIDATA{OutputFilter=late$x2$.dll}
%TCIDATA{Version=5.00.0.2552}
%TCIDATA{CSTFile=40 LaTeX article.cst}
%TCIDATA{Created=Thursday, August 21, 2008 14:03:59}
%TCIDATA{LastRevised=Wednesday, October 01, 2014 12:46:33}
%TCIDATA{<META NAME="GraphicsSave" CONTENT="32">}
%TCIDATA{<META NAME="SaveForMode" CONTENT="1">}
%TCIDATA{<META NAME="DocumentShell" CONTENT="Standard LaTeX\Blank - Standard LaTeX Article">}
%TCIDATA{Language=American English}
\newtheorem{theorem}{Theorem}
\newtheorem{acknowledgement}[theorem]{Acknowledgement}
\newtheorem{algorithm}[theorem]{Algorithm}
\newtheorem{axiom}[theorem]{Axiom}
\newtheorem{case}[theorem]{Case}
\newtheorem{claim}[theorem]{Claim}
\newtheorem{conclusion}[theorem]{Conclusion}
\newtheorem{condition}[theorem]{Condition}
\newtheorem{conjecture}[theorem]{Conjecture}
\newtheorem{corollary}[theorem]{Corollary}
\newtheorem{criterion}[theorem]{Criterion}
\newtheorem{definition}[theorem]{Definition}
\newtheorem{example}[theorem]{Example}
\newtheorem{exercise}[theorem]{Exercise}
\newtheorem{lemma}[theorem]{Lemma}
\newtheorem{notation}[theorem]{Notation}
\newtheorem{problem}[theorem]{Problem}
\newtheorem{proposition}[theorem]{Proposition}
\newtheorem{remark}[theorem]{Remark}
\newtheorem{solution}[theorem]{Solution}
\newtheorem{summary}[theorem]{Summary}
\newenvironment{proof}[1][Proof]{\noindent\textbf{#1.} }{\ \rule{0.5em}{0.5em}}

\usepackage{fancyhdr}
\setlength\headheight{26pt}
\pagestyle{fancy}
\lhead{{\footnotesize Sleeping Barber Report}}
\rhead{{\footnotesize Christopher Chapline}}
\begin{document}

\section*{Compilation and Running}

To compile the project, run the following command:

\lstset{language=Bash}
\begin{lstlisting}[frame=single]
make barber
\end{lstlisting}\\
\\
The program takes the following command line arguments:
\begin{itemize}
    \begin{item}
        \textbf{numberOfCustomers}: The number of customers that will arrive throughout the program
    \end{item}

    \begin{item}
        \textbf{haircutTime}: The number of seconds that it will take for a customer's haircut to be completed
    \end{item}

    \begin{item}
        \textbf{arrivalInterval}: The number of seconds in between the arrival of the customers (where the 0th customer arrives immediately)
    \end{item}
\end{itemize}
\\
An example of running this program follows:
\begin{verbatim}
java Barbers 3 5 1
The barber shop has opened for the day! Serving 3 customers today!
Customer 0 has arrived at the barber shop!
Customer 0 has started their haircut!
Customer 1 has arrived at the barber shop!
Customer 2 has arrived at the barber shop!
Customer 0 has finished their haircut!
Customer 0 has left the building!
Customer 1 has started their haircut!
Customer 1 has finished their haircut!
Customer 1 has left the building!
Customer 2 has started their haircut!
Customer 2 has finished their haircut!
Customer 2 has left the building!
The barber shop is closing for the day!
\end{verbatim}

\end{document}

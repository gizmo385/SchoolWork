\documentclass{article}%
\usepackage{amsmath}
\usepackage{amsfonts}
\usepackage{amssymb}
\usepackage{graphicx}
\usepackage{tikz}
\usepackage{hyperref}%
\setcounter{MaxMatrixCols}{30}
%TCIDATA{OutputFilter=latex2.dll}
%TCIDATA{Version=5.00.0.2552}
%TCIDATA{CSTFile=40 LaTeX article.cst}
%TCIDATA{Created=Thursday, August 21, 2008 14:03:59}
%TCIDATA{LastRevised=Wednesday, October 01, 2014 12:46:33}
%TCIDATA{<META NAME="GraphicsSave" CONTENT="32">}
%TCIDATA{<META NAME="SaveForMode" CONTENT="1">}
%TCIDATA{<META NAME="DocumentShell" CONTENT="Standard LaTeX\Blank - Standard LaTeX Article">}
%TCIDATA{Language=American English}
\newtheorem{theorem}{Theorem}
\newtheorem{acknowledgement}[theorem]{Acknowledgement}
\newtheorem{algorithm}[theorem]{Algorithm}
\newtheorem{axiom}[theorem]{Axiom}
\newtheorem{case}[theorem]{Case}
\newtheorem{claim}[theorem]{Claim}
\newtheorem{conclusion}[theorem]{Conclusion}
\newtheorem{condition}[theorem]{Condition}
\newtheorem{conjecture}[theorem]{Conjecture}
\newtheorem{corollary}[theorem]{Corollary}
\newtheorem{criterion}[theorem]{Criterion}
\newtheorem{definition}[theorem]{Definition}
\newtheorem{example}[theorem]{Example}
\newtheorem{exercise}[theorem]{Exercise}
\newtheorem{lemma}[theorem]{Lemma}
\newtheorem{notation}[theorem]{Notation}
\newtheorem{problem}[theorem]{Problem}
\newtheorem{proposition}[theorem]{Proposition}
\newtheorem{remark}[theorem]{Remark}
\newtheorem{solution}[theorem]{Solution}
\newtheorem{summary}[theorem]{Summary}
\newenvironment{proof}[1][Proof]{\noindent\textbf{#1.} }{\ \rule{0.5em}{0.5em}}
\begin{document}

\title{CS 352 - Homework 1}
\author{Christopher Chapline}
\maketitle

\section{Problem 1}

\subsection{Part a}
In the case that $x=5$ the program will terminate. This is because upon start, it will execute $x-2$ and $x-3$ in some order, which will complete the first and second arms of the process. Then, since $x=0$, it will activate and terminate the third arm, which leaves the final value of $x$ to be 0.

\noindent It will also terminate for $x=5$. In this case, the third arm will immediately activate, followed by the first and second arm in some order. In this second case, $x=0$ will be the final value.

\subsection{Part b}
The program might terminate for the values $x=2$ and $x=3$. In the first case, if the first arm executes, then the third arm, and finally the second arm; the program will terminate with a final value of $x=3$. Similarly, when $x=3$, if the second arm executes, then the third arm, then the first arm; then the program will terminate with a final value of $x=2$.


\section{Problem 2}

\subsection{Part a}

$\forall i : 1 \leq i \leq m, \neg \exists j : 1 \leq j \leq n : a[i] \geq b[j]$

\subsection{Part c}

$\neg((\exists i : 1 \leq i \leq m : a[i] = 0) \wedge (\exists j : 1 \leq j \leq m : b[j] = 0))$

\subsection{Part e}

$\forall i : 1 \leq i \leq m, \exists j : 1 \leq j \leq n : a[i] = b[j]$

\section{Problem 3}

\subsection{Part a}

While it is possible that it could stop, it is highly unlikely that it will. For it to stop, the scheduler must stop the process executing the while loop immediately when $x=10$ and it must start the process that checks for $x==0$ before it restarts the while loop process. While possible, this is improbable

\subsection{Part b}

In the case that the scheduling is strongly fair, the program would not terminate. The first process would not get a chance to run because it is not infinitely often true.

\subsection{Part c.a}
The addition of the third arm would cause the program to eventually terminate. This is because whenever the $x$ value fell below 0, the third arm would automatically reset its value to be 10. Eventually, it will fall into the case where it stops at $x=0$ and $c$ gets set to false.

\subsection{Part c.b}

Similarly to c.a, this would eventually terminate because it would fall into the $x=0$ case.

\end{document}

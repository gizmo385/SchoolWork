\documentclass{article}%
\usepackage{amsmath}
\usepackage{amsfonts}
\usepackage{amssymb}
\usepackage{listings}
\usepackage{graphicx}
\usepackage{tikz}
\usepackage{hyperref}%
\usepackage[a4paper,includeheadfoot,margin=0.5in]{geometry}
\setcounter{MaxMatrixCols}{30}
%TCIDATA{OutputFilter=late$x2$.dll}
%TCIDATA{Version=5.00.0.2552}
%TCIDATA{CSTFile=40 LaTeX article.cst}
%TCIDATA{Created=Thursday, August 21, 2008 14:03:59}
%TCIDATA{LastRevised=Wednesday, October 01, 2014 12:46:33}
%TCIDATA{<META NAME="GraphicsSave" CONTENT="32">}
%TCIDATA{<META NAME="SaveForMode" CONTENT="1">}
%TCIDATA{<META NAME="DocumentShell" CONTENT="Standard LaTeX\Blank - Standard LaTeX Article">}
%TCIDATA{Language=American English}
\newtheorem{theorem}{Theorem}
\newtheorem{acknowledgement}[theorem]{Acknowledgement}
\newtheorem{algorithm}[theorem]{Algorithm}
\newtheorem{axiom}[theorem]{Axiom}
\newtheorem{case}[theorem]{Case}
\newtheorem{claim}[theorem]{Claim}
\newtheorem{conclusion}[theorem]{Conclusion}
\newtheorem{condition}[theorem]{Condition}
\newtheorem{conjecture}[theorem]{Conjecture}
\newtheorem{corollary}[theorem]{Corollary}
\newtheorem{criterion}[theorem]{Criterion}
\newtheorem{definition}[theorem]{Definition}
\newtheorem{example}[theorem]{Example}
\newtheorem{exercise}[theorem]{Exercise}
\newtheorem{lemma}[theorem]{Lemma}
\newtheorem{notation}[theorem]{Notation}
\newtheorem{problem}[theorem]{Problem}
\newtheorem{proposition}[theorem]{Proposition}
\newtheorem{remark}[theorem]{Remark}
\newtheorem{solution}[theorem]{Solution}
\newtheorem{summary}[theorem]{Summary}
\newenvironment{proof}[1][Proof]{\noindent\textbf{#1.} }{\ \rule{0.5em}{0.5em}}

\usepackage{fancyhdr}
\setlength\headheight{26pt}
\pagestyle{fancy}
\lhead{{\footnotesize Assignment 9}}
\rhead{{\footnotesize Christopher Chapline}}
\begin{document}

\section*{Problem 1}
%Give the truth table for ((p → q) ↔ (¬q → ¬p))

\begin{tabular}{| l | l | l | l | l | l | l |}
    \hline
    $p$ & $q$ & $(p \rightarrow q)$ & $\neg q$ & $\neg p$ & $(\neg q \rightarrow \neg p)$ & $((p \rightarrow q) \leftrightarrow (\neg q \rightarrow \neg p))$ \\ \hline
    $T$ & $T$ & $T$                 & $F$      & $F$      & $T$                           & $T$ \\ \hline
    $T$ & $F$ & $F$                 & $T$      & $F$      & $F$                           & $T$ \\ \hline
    $F$ & $T$ & $T$                 & $F$      & $T$      & $T$                           & $T$ \\ \hline
    $F$ & $F$ & $T$                 & $T$      & $T$      & $T$                           & $T$ \\ \hline
\end{tabular}
\section*{Problem 2}
Consider the following set of worlds:\\
\\
$W_1 = \{r\}$\\
\\
$W_2 = \{\neg r\}$\\
\\
In this configuration, in $W_1$, both $\Diamond r$ and $\Diamond \neg r$ are true. In $W_2$, both $\Diamond r$ and $\Diamond \neg r$ are also
true. Thus access restrictions are not necessary.
\section*{Problem 3}
Yes it is true. $F(b)$ is false according to our model. By the definition of implication, a false antecedent yields a true expression
regardless of the value of consequent.

\section*{Problem 4}
It is not true. For example, let $x = a$ and let $y = b$. As is shown in the model, $G(a, b)$ is false.

\end{document}

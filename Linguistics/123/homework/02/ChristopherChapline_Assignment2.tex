\documentclass{article}%
\usepackage{amsmath}
\usepackage{amsfonts}
\usepackage{amssymb}
\usepackage{graphicx}
\usepackage{listings}
\usepackage{tikz}
\usepackage{hyperref}%
\usepackage[a4paper,includeheadfoot,margin=0.5in]{geometry}
\setcounter{MaxMatrixCols}{30}
%TCIDATA{OutputFilter=late$x2$.dll}
%TCIDATA{Version=5.00.0.2552}
%TCIDATA{CSTFile=40 LaTeX article.cst}
%TCIDATA{Created=Thursday, August 21, 2008 14:03:59}
%TCIDATA{LastRevised=Wednesday, October 01, 2014 12:46:33}
%TCIDATA{<META NAME="GraphicsSave" CONTENT="32">}
%TCIDATA{<META NAME="SaveForMode" CONTENT="1">}
%TCIDATA{<META NAME="DocumentShell" CONTENT="Standard LaTeX\Blank - Standard LaTeX Article">}
%TCIDATA{Language=American English}
\newtheorem{theorem}{Theorem}
\newtheorem{acknowledgement}[theorem]{Acknowledgement}
\newtheorem{algorithm}[theorem]{Algorithm}
\newtheorem{axiom}[theorem]{Axiom}
\newtheorem{case}[theorem]{Case}
\newtheorem{claim}[theorem]{Claim}
\newtheorem{conclusion}[theorem]{Conclusion}
\newtheorem{condition}[theorem]{Condition}
\newtheorem{conjecture}[theorem]{Conjecture}
\newtheorem{corollary}[theorem]{Corollary}
\newtheorem{criterion}[theorem]{Criterion}
\newtheorem{definition}[theorem]{Definition}
\newtheorem{example}[theorem]{Example}
\newtheorem{exercise}[theorem]{Exercise}
\newtheorem{lemma}[theorem]{Lemma}
\newtheorem{notation}[theorem]{Notation}
\newtheorem{problem}[theorem]{Problem}
\newtheorem{proposition}[theorem]{Proposition}
\newtheorem{remark}[theorem]{Remark}
\newtheorem{solution}[theorem]{Solution}
\newtheorem{summary}[theorem]{Summary}
\newenvironment{proof}[1][Proof]{\noindent\textbf{#1.} }{\ \rule{0.5em}{0.5em}}

\usepackage{fancyhdr}
\setlength\headheight{26pt}
\pagestyle{fancy}
\lhead{{\footnotesize Assignment 2}}
\rhead{{\footnotesize Christopher Chapline}}
\begin{document}

\section*{Problem 1}
%We’ve cited several structures in English that show that the set of words
%and the set of sentences are infinite. Find another that’s different from
%the ones cited in the text.

A structure would could be used to prove that the set of English sentences is infinite could involve repeated conjunction. Consider the following example:
\lstset{language=Bash}
\begin{lstlisting}[frame=single]
This is okay.
This and this are okay.
This and this and this are okay.
...
\end{lstlisting}\\
\\
Since one could repeatedly add "and this" to that sentence, one could create an infinite number of sentences of that form.

\vspace{10mm}

\noindent A similar structure could be used to show that infinitude of English words. We can prepend and append prefixes and suffixes to words
\textit{ad infinum} to create an infinite number of words. Consider the following example:
\begin{lstlisting}[frame=single]
terror
terrorize
terrorization
terrorizational
deterrorizational
...
\end{lstlisting}\\
\\
... and so on.

\section*{Problem 2}

Since a set contains no duplicates, we can say that $| A |$ is 1.

\section*{Problem 3}

For sets $A$ and $B$, $A \subseteq B$ implies that all elements in $A$ are also elements in $B$. Since $\emptyset$ contains no elements,
then it is trivial, for any set $C$, to say $\emptyset \subseteq C$.

\end{document}

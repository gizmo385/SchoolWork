\documentclass{article}%
\usepackage{amsmath}
\usepackage{amsfonts}
\usepackage{amssymb}
\usepackage{graphicx}
\usepackage{listings}
\usepackage{tikz}
\usepackage{hyperref}%
\usepackage[a4paper,includeheadfoot,margin=0.5in]{geometry}
\setcounter{MaxMatrixCols}{30}
%TCIDATA{OutputFilter=late$x2$.dll}
%TCIDATA{Version=5.00.0.2552}
%TCIDATA{CSTFile=40 LaTeX article.cst}
%TCIDATA{Created=Thursday, August 21, 2008 14:03:59}
%TCIDATA{LastRevised=Wednesday, October 01, 2014 12:46:33}
%TCIDATA{<META NAME="GraphicsSave" CONTENT="32">}
%TCIDATA{<META NAME="SaveForMode" CONTENT="1">}
%TCIDATA{<META NAME="DocumentShell" CONTENT="Standard LaTeX\Blank - Standard LaTeX Article">}
%TCIDATA{Language=American English}
\newtheorem{theorem}{Theorem}
\newtheorem{acknowledgement}[theorem]{Acknowledgement}
\newtheorem{algorithm}[theorem]{Algorithm}
\newtheorem{axiom}[theorem]{Axiom}
\newtheorem{case}[theorem]{Case}
\newtheorem{claim}[theorem]{Claim}
\newtheorem{conclusion}[theorem]{Conclusion}
\newtheorem{condition}[theorem]{Condition}
\newtheorem{conjecture}[theorem]{Conjecture}
\newtheorem{corollary}[theorem]{Corollary}
\newtheorem{criterion}[theorem]{Criterion}
\newtheorem{definition}[theorem]{Definition}
\newtheorem{example}[theorem]{Example}
\newtheorem{exercise}[theorem]{Exercise}
\newtheorem{lemma}[theorem]{Lemma}
\newtheorem{notation}[theorem]{Notation}
\newtheorem{problem}[theorem]{Problem}
\newtheorem{proposition}[theorem]{Proposition}
\newtheorem{remark}[theorem]{Remark}
\newtheorem{solution}[theorem]{Solution}
\newtheorem{summary}[theorem]{Summary}
\newenvironment{proof}[1][Proof]{\noindent\textbf{#1.} }{\ \rule{0.5em}{0.5em}}

\usepackage{fancyhdr}
\setlength\headheight{26pt}
\pagestyle{fancy}
\lhead{{\footnotesize Assignment 3}}
\rhead{{\footnotesize Christopher Chapline}}
\begin{document}

\section*{Problem 1}

%As we noted on page 34, it follows that if A ⊆ B and B ⊆ A, that
%A = B. Explain why this is so. Prove that this is so.

The statement that $A \subseteq B$ implies that every element in $A$ is also contained within $B$. Additionally, the statement
$B \subseteq A$ implies that every element in $B$ is also contained within $A$. If $A$ contains all of the elements in $B$, and
$B$ contains all of the elements in $A$, then it follows (as night follows day), that $A$ and $B$ must contain the same elements.

\section*{Problem 2}

%The empty set is a proper subset of all sets except one. Which one and
%why?

The empty set is not a proper subset of itself. The definition of proper subset requires that the second operand contain elements not
contained within the first operand. This would imply that no set could be a proper subset of itself, because it cannot possibly meet
the requirement for additional elements to be present. This property holds for the empty set as well.

\section*{Problem 3}

%We gave a number of theorems on page 38. Prove these
%1. If A ⊆ B then A ∪ C ⊆ B ∪ C
%2. If A ⊆ B then A ∩ C ⊆ B ∩ C
%3. If A ⊆ B then C − B ⊆ C − A
%4. A ∩ (B − A) = ∅

\begin{theorem}
    If $A \subseteq B$, then $A \bigcup C \subseteq B \bigcup C$.
\end{theorem}

\begin{proof}
\end{proof}

\begin{theorem}
    If $A \subseteq B$, then $A \bigcap C \subseteq B \bigcap C$.
\end{theorem}

\begin{proof}
\end{proof}

\begin{theorem}
    If $A \subseteq B$, then $A - B \subseteq C - A$
\end{theorem}

\begin{proof}
\end{proof}

\begin{theorem}
    $A \bigcap (B - A) = \emptyset$
\end{theorem}

\begin{proof}
\end{proof}

\end{document}

\documentclass{article}%
\usepackage{amsmath}
\usepackage{amsfonts}
\usepackage{amssymb}
\usepackage{listings}
\usepackage{graphicx}
\usepackage{tikz}
\usepackage{hyperref}%
\usepackage[a4paper,includeheadfoot,margin=0.5in]{geometry}
\setcounter{MaxMatrixCols}{30}
%TCIDATA{OutputFilter=late$x2$.dll}
%TCIDATA{Version=5.00.0.2552}
%TCIDATA{CSTFile=40 LaTeX article.cst}
%TCIDATA{Created=Thursday, August 21, 2008 14:03:59}
%TCIDATA{LastRevised=Wednesday, October 01, 2014 12:46:33}
%TCIDATA{<META NAME="GraphicsSave" CONTENT="32">}
%TCIDATA{<META NAME="SaveForMode" CONTENT="1">}
%TCIDATA{<META NAME="DocumentShell" CONTENT="Standard LaTeX\Blank - Standard LaTeX Article">}
%TCIDATA{Language=American English}
\newtheorem{theorem}{Theorem}
\newtheorem{acknowledgement}[theorem]{Acknowledgement}
\newtheorem{algorithm}[theorem]{Algorithm}
\newtheorem{axiom}[theorem]{Axiom}
\newtheorem{case}[theorem]{Case}
\newtheorem{claim}[theorem]{Claim}
\newtheorem{conclusion}[theorem]{Conclusion}
\newtheorem{condition}[theorem]{Condition}
\newtheorem{conjecture}[theorem]{Conjecture}
\newtheorem{corollary}[theorem]{Corollary}
\newtheorem{criterion}[theorem]{Criterion}
\newtheorem{definition}[theorem]{Definition}
\newtheorem{example}[theorem]{Example}
\newtheorem{exercise}[theorem]{Exercise}
\newtheorem{lemma}[theorem]{Lemma}
\newtheorem{notation}[theorem]{Notation}
\newtheorem{problem}[theorem]{Problem}
\newtheorem{proposition}[theorem]{Proposition}
\newtheorem{remark}[theorem]{Remark}
\newtheorem{solution}[theorem]{Solution}
\newtheorem{summary}[theorem]{Summary}
\newenvironment{proof}[1][Proof]{\noindent\textbf{#1.} }{\ \rule{0.5em}{0.5em}}

\usepackage{fancyhdr}
\setlength\headheight{26pt}
\pagestyle{fancy}
\lhead{{\footnotesize Assignment 10}}
\rhead{{\footnotesize Christopher Chapline}}
\begin{document}

\section*{Problem 1}
\begin{tabular}{| l | l | l | l | l |}
    \hline
    $p$ & $\neg p$  & $(p \rightarrow \neg p)$  & $(\neg p \rightarrow p)$  & $((p \rightarrow \neg p) \vee (\neg p \rightarrow p)$ \\ \hline
    $T$ & $F$       & $F$                       & $T$                       & $T$ \\ \hline
    $F$ & $T$       & $T$                       & $F$                       & $T$ \\ \hline
\end{tabular}

\section*{Problem 2}

A tautology is a proposition which is true for any combination of truth values. An example from modal logic would
be the following:\\
\\
Let $w_1 = \{p\}$ and $w_2 = \{p\}$. Thus, $\Box p$ is tautology in this model.

\section*{Problem 3}

Yes. For all entities in the model, $\{a, b, c\}$, $G(x, x)$ is true.

\section*{Problem 4}

$G(a, a) \wedge G(b, b) \wedge G(c, c)$.

\end{document}

\documentclass{article}%
\usepackage{amsmath}
\usepackage{amsfonts}
\usepackage{amssymb}
\usepackage{listings}
\usepackage{graphicx}
\usepackage{tikz}
\usepackage{hyperref}%
\usepackage[a4paper,includeheadfoot,margin=0.5in]{geometry}
\setcounter{MaxMatrixCols}{30}
%TCIDATA{OutputFilter=late$x2$.dll}
%TCIDATA{Version=5.00.0.2552}
%TCIDATA{CSTFile=40 LaTeX article.cst}
%TCIDATA{Created=Thursday, August 21, 2008 14:03:59}
%TCIDATA{LastRevised=Wednesday, October 01, 2014 12:46:33}
%TCIDATA{<META NAME="GraphicsSave" CONTENT="32">}
%TCIDATA{<META NAME="SaveForMode" CONTENT="1">}
%TCIDATA{<META NAME="DocumentShell" CONTENT="Standard LaTeX\Blank - Standard LaTeX Article">}
%TCIDATA{Language=American English}
\newtheorem{theorem}{Theorem}
\newtheorem{acknowledgement}[theorem]{Acknowledgement}
\newtheorem{algorithm}[theorem]{Algorithm}
\newtheorem{axiom}[theorem]{Axiom}
\newtheorem{case}[theorem]{Case}
\newtheorem{claim}[theorem]{Claim}
\newtheorem{conclusion}[theorem]{Conclusion}
\newtheorem{condition}[theorem]{Condition}
\newtheorem{conjecture}[theorem]{Conjecture}
\newtheorem{corollary}[theorem]{Corollary}
\newtheorem{criterion}[theorem]{Criterion}
\newtheorem{definition}[theorem]{Definition}
\newtheorem{example}[theorem]{Example}
\newtheorem{exercise}[theorem]{Exercise}
\newtheorem{lemma}[theorem]{Lemma}
\newtheorem{notation}[theorem]{Notation}
\newtheorem{problem}[theorem]{Problem}
\newtheorem{proposition}[theorem]{Proposition}
\newtheorem{remark}[theorem]{Remark}
\newtheorem{solution}[theorem]{Solution}
\newtheorem{summary}[theorem]{Summary}
\newenvironment{proof}[1][Proof]{\noindent\textbf{#1.} }{\ \rule{0.5em}{0.5em}}

\usepackage{fancyhdr}
\setlength\headheight{26pt}
\pagestyle{fancy}
\lhead{{\footnotesize Assignment 5}}
\rhead{{\footnotesize Christopher Chapline}}
\begin{document}

\section*{Problem 1}

\begin{tabular}{l | l | l | l |}
    $p$ & $q$ & $(p \rightarrow q)$ & $(p \rightarrow (p \rightarrow q))$ \\ \hline
    $T$ & $T$ & $T$                 & $T$ \\ \hline
    $T$ & $F$ & $F$                 & $F$ \\ \hline
    $F$ & $T$ & $T$                 & $T$ \\ \hline
    $F$ & $F$ & $T$                 & $T$ \\ \hline
\end{tabular}\\
\\
\vspace{3mm}

\noindent
\begin{tabular}{l | l | l | l |}
    $p$ & $(p \rightarrow p)$   & $q$ & $((p \rightarrow p) \rightarrow q)$ \\ \hline
    $T$ & $T$                   & $T$ & $T$ \\ \hline
    $T$ & $T$                   & $F$ & $F$ \\ \hline
    $F$ & $T$                   & $T$ & $T$ \\ \hline
    $F$ & $T$                   & $F$ & $F$ \\ \hline
\end{tabular}
\\
\vspace{3mm}

\noindent
\begin{tabular}{l | l | l | l | l |}
    $p$ & $q$ & $\neg q$    & $\neg \neg q$ & $(p \leftrightarrow \neg \neg q)$ \\ \hline
    $T$ & $T$ & $F$         & $T$           & $T$ \\ \hline
    $T$ & $F$ & $T$         & $F$           & $F$ \\ \hline
    $F$ & $T$ & $F$         & $T$           & $F$ \\ \hline
    $F$ & $F$ & $T$         & $F$           & $T$ \\ \hline
\end{tabular}
\\
\vspace{3mm}

\noindent
\begin{tabular}{l | l | l | l |}
    $p$ & $\neg p$  & $(p \vee \neg p)$ & $\neg (p \vee \neg p)$ \\ \hline
    $T$ & $F$       & $T$               & $F$ \\ \hline
    $T$ & $F$       & $T$               & $F$ \\ \hline
    $F$ & $T$       & $T$               & $F$ \\ \hline
    $F$ & $T$       & $T$               & $F$ \\ \hline
\end{tabular}

\section*{Problem 2}

The most obvious WFF that would produce this truth table would be $(p \rightarrow q)$. An
second WFF which produces the truth table would be $(\neg p \vee q)$.

\section*{Problem 3}

An example of a tautologous WFF would be $(p \vee \neg p)$. In this WFF, regardless of if
$p$ is true or false, the WFF will be true.

\end{document}

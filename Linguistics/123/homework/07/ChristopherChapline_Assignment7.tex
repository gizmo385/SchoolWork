\documentclass{article}%
\usepackage{amsmath}
\usepackage{amsfonts}
\usepackage{amssymb}
\usepackage{listings}
\usepackage{graphicx}
\usepackage{tikz}
\usepackage{hyperref}%
\usepackage[a4paper,includeheadfoot,margin=0.5in]{geometry}
\setcounter{MaxMatrixCols}{30}
%TCIDATA{OutputFilter=late$x2$.dll}
%TCIDATA{Version=5.00.0.2552}
%TCIDATA{CSTFile=40 LaTeX article.cst}
%TCIDATA{Created=Thursday, August 21, 2008 14:03:59}
%TCIDATA{LastRevised=Wednesday, October 01, 2014 12:46:33}
%TCIDATA{<META NAME="GraphicsSave" CONTENT="32">}
%TCIDATA{<META NAME="SaveForMode" CONTENT="1">}
%TCIDATA{<META NAME="DocumentShell" CONTENT="Standard LaTeX\Blank - Standard LaTeX Article">}
%TCIDATA{Language=American English}
\newtheorem{theorem}{Theorem}
\newtheorem{acknowledgement}[theorem]{Acknowledgement}
\newtheorem{algorithm}[theorem]{Algorithm}
\newtheorem{axiom}[theorem]{Axiom}
\newtheorem{case}[theorem]{Case}
\newtheorem{claim}[theorem]{Claim}
\newtheorem{conclusion}[theorem]{Conclusion}
\newtheorem{condition}[theorem]{Condition}
\newtheorem{conjecture}[theorem]{Conjecture}
\newtheorem{corollary}[theorem]{Corollary}
\newtheorem{criterion}[theorem]{Criterion}
\newtheorem{definition}[theorem]{Definition}
\newtheorem{example}[theorem]{Example}
\newtheorem{exercise}[theorem]{Exercise}
\newtheorem{lemma}[theorem]{Lemma}
\newtheorem{notation}[theorem]{Notation}
\newtheorem{problem}[theorem]{Problem}
\newtheorem{proposition}[theorem]{Proposition}
\newtheorem{remark}[theorem]{Remark}
\newtheorem{solution}[theorem]{Solution}
\newtheorem{summary}[theorem]{Summary}
\newenvironment{proof}[1][Proof]{\noindent\textbf{#1.} }{\ \rule{0.5em}{0.5em}}

\usepackage{fancyhdr}
\setlength\headheight{26pt}
\pagestyle{fancy}
\lhead{{\footnotesize Assignment 7}}
\rhead{{\footnotesize Christopher Chapline}}
\begin{document}

\section*{Problem 1}
Prove $r$ from $(p \wedge \neg p)$.\\
\\
\begin{tabular}{l l | l}
    1 & $(p \wedge \neg p)$ & Premise \\ \hline
    2 & $\neg r$ & Proof by contradiction \\ \hline
    3 & $p$ & 1, Simp. \\ \hline
    4 & $(\neg p \wedge p)$ & 1, Comm. \\ \hline
    5 & $\neg p$ & 4, Simp. \\ \hline
    6 & $r$ & 3, 5 Contradiction
\end{tabular}

\section*{Problem 2}
Prove $\neg r$ from $(p \wedge \neg p)$.\\
\\
\begin{tabular}{l l | l}
    1 & $(p \wedge \neg p)$ & Premise \\ \hline
    2 & $r$ & Proof by contradiction \\ \hline
    3 & $p$ & 1, Simp. \\ \hline
    4 & $(\neg p \wedge p)$ & 1, Comm. \\ \hline
    5 & $\neg p$ & 4, Simp. \\ \hline
    6 & $\neg r$ & 3, 5 Contradiction
\end{tabular}

\section*{Problem 3}
Draw a configuration of worlds where $(\neg p \wedge \Box \Diamond p)$ and explain clearly why it works.\\
\\
One set of worlds might look like this:\\
\\
$W_1 = \{\neg p\}$\\
\\
$W_2 = \{p\}$\\
\\
In this configuration, one world, $W_1$ satisfies the $\neg p$. The second world, $W_2$ satisfies that it be necessary
that $p$ is possible.

\section*{Problem 4}
$(\Box p \rightarrow \Diamond p)$ is a tautology. Explain clearly why.\\
\\
If you have a configuration of worlds such that $\Box p$ is true, then each world in the configuration has $p$. Thus,
$\Diamond p$ would be true for this configuration. If the configuration of worlds does not satisfy $\Box p$, then the
implication is satisfied regardless of $\Diamond p$.

\end{document}

\documentclass{article}%
\usepackage{amsmath}
\usepackage{amsfonts}
\usepackage{amssymb}
\usepackage{graphicx}
\usepackage{tikz}
\usepackage{hyperref}%
\setcounter{MaxMatrixCols}{30}
%TCIDATA{OutputFilter=late$x2$.dll}
%TCIDATA{Version=5.00.0.2552}
%TCIDATA{CSTFile=40 LaTeX article.cst}
%TCIDATA{Created=Thursday, August 21, 2008 14:03:59}
%TCIDATA{LastRevised=Wednesday, October 01, 2014 12:46:33}
%TCIDATA{<META NAME="GraphicsSave" CONTENT="32">}
%TCIDATA{<META NAME="SaveForMode" CONTENT="1">}
%TCIDATA{<META NAME="DocumentShell" CONTENT="Standard LaTeX\Blank - Standard LaTeX Article">}
%TCIDATA{Language=American English}
\newtheorem{theorem}{Theorem}
\newtheorem{acknowledgement}[theorem]{Acknowledgement}
\newtheorem{algorithm}[theorem]{Algorithm}
\newtheorem{axiom}[theorem]{Axiom}
\newtheorem{case}[theorem]{Case}
\newtheorem{claim}[theorem]{Claim}
\newtheorem{conclusion}[theorem]{Conclusion}
\newtheorem{condition}[theorem]{Condition}
\newtheorem{conjecture}[theorem]{Conjecture}
\newtheorem{corollary}[theorem]{Corollary}
\newtheorem{criterion}[theorem]{Criterion}
\newtheorem{definition}[theorem]{Definition}
\newtheorem{example}[theorem]{Example}
\newtheorem{exercise}[theorem]{Exercise}
\newtheorem{lemma}[theorem]{Lemma}
\newtheorem{notation}[theorem]{Notation}
\newtheorem{problem}[theorem]{Problem}
\newtheorem{proposition}[theorem]{Proposition}
\newtheorem{remark}[theorem]{Remark}
\newtheorem{solution}[theorem]{Solution}
\newtheorem{summary}[theorem]{Summary}
\newenvironment{proof}[1][Proof]{\noindent\textbf{#1.} }{\ \rule{0.5em}{0.5em}}

\usepackage{fancyhdr}
\setlength\headheight{26pt}
\pagestyle{fancy}
\lhead{{\footnotesize Math 215 - Homework 11}}
\rhead{{\footnotesize Christopher Chapline}}
\begin{document}

\section*{Chapter 4.6}

\subsection*{Problem 8}

The dimension of the null space of $A$ ($dim(Null(A))$), would be 4.\\
\\
The column space of $A$ would not be equal $\mathbb{R}^4$. The vectors in $Coll(A)$ are in $\mathbb{R}^6$ and could not span $\mathbb{R}^4$.

\subsection*{Problem 10}

By Rank-Nullity, dimension of the column space of is 7 (the number of columns in the matrix) minus 5 (the dimension of the null space), which is 2. Thus, $dim(coll(A)) = 2$.

\subsection*{Problem 12}

The dimension of the row space of a matrix is equivalent to the rank of the matrix. The rank of $A$ is 4, thus the dimension of the row space of $A$ is also 4.

\section*{Chapter 4.7}

\subsection*{Problem 8}

Let $\vec{b_1} =
\begin{bmatrix}
    -1 \\ 8
\end{bmatrix},
\vec{b_2} =
\begin{bmatrix}
    1 \\ -7
\end{bmatrix},
\vec{c_1} =
\begin{bmatrix}
    1 \\ 2
\end{bmatrix},
\vec{c_2} =
\begin{bmatrix}
    1 \\ 1
\end{bmatrix}$.\\
\\[0.1in]
To find $P_{C \leftarrow B}$, we row reduce this matrix until the left side is equal to the identity matrix:\\
\\
$
\begin{bmatrix}
    1 & 1 & \vline & 1 & 1 \\
    2 & 1 & \vline & 8 & -7
\end{bmatrix}
\backsim
\begin{bmatrix}
    1 & 1 & \vline & 1 & 1 \\
    0 & 1 & \vline & -10 & 9
\end{bmatrix}
\backsim
\begin{bmatrix}
    1 & 0 & \vline & 9 & -9 \\
    0 & 1 & \vline & -10 & 9
\end{bmatrix}$\\
\\[0.05in]
Thus, $P_{C \leftarrow B} =
\begin{bmatrix}
    9 & -9 \\
    -10 & 9
\end{bmatrix}$.\\
\\[0.1in]
There are two ways we can compute $P_{B \leftarrow C}$. We can either compute it using the same process we used to compute $P_{C \leftarrow B}$ or we can find the inverse
of $P_{C \leftarrow B}$. Since $P_{C \leftarrow B}$ is a 2x2 matrix, its inverse has a simple closed form:\\
\\
$
\begin{bmatrix}
    a & b \\ c & d
\end{bmatrix}^{-1}
=
\frac{1}{ad - bc}
\begin{bmatrix}
    d & -b \\ -c & a
\end{bmatrix}
\Rightarrow
\frac{1}{(9)(9) - (-9)(-10)}
\begin{bmatrix}
    9 & 9 \\ 10 & 9
\end{bmatrix}
=
-\frac{1}{9}
\begin{bmatrix}
    9 & 9 \\ 10 & 9
\end{bmatrix}$\\
\\[0.1in]
Thus, $P_{B \leftarrow C} =
-\frac{1}{9}
\begin{bmatrix}
    9 & 9 \\ 10 & 9
\end{bmatrix}$.\\

\subsection*{Problem 10}

Let $\vec{b_1} =
\begin{bmatrix}
    6 \\ -12
\end{bmatrix},
\vec{b_2} =
\begin{bmatrix}
    4 \\ 2
\end{bmatrix},
\vec{c_1} =
\begin{bmatrix}
    4 \\ 2
\end{bmatrix},
\vec{c_2} =
\begin{bmatrix}
    3 \\ 9
\end{bmatrix}$.\\
\\[0.1in]
To find $P_{B \leftarrow C}$, we row reduce this matrix until the left side is equal to the identity matrix:\\
\\
$
\begin{bmatrix}
    6 & 4 & \vline & 4 & 3 \\
    -12 & 2 & \vline & 2 & 9
\end{bmatrix}
\backsim
\begin{bmatrix}
    6 & 4 & \vline & 4 & 3 \\
    0 & 10 & \vline & 10 & 15
\end{bmatrix}
\backsim
\begin{bmatrix}
    6 & 4 & \vline & 4 & 3 \\
    0 & 2 & \vline & 2 & 3
\end{bmatrix}
\backsim\\
\\[0.05in]
\begin{bmatrix}
    6 & 0 & \vline & 0 & -3 \\
    0 & 2 & \vline & 2 & 3
\end{bmatrix}
\backsim
\begin{bmatrix}
    1 & 0 & \vline & 0 & -\frac{1}{2} \\
    0 & 1 & \vline & 1 & \frac{3}{2}
\end{bmatrix}$\\
\\[0.1in]
Therefore, $P_{B \leftarrow C} =
\begin{bmatrix}
    0 & -\frac{1}{2} \\
    1 & \frac{3}{2}
\end{bmatrix}$. Once again, we can find $P_{C \leftarrow B}$ by find the inverse of $P_{B \leftarrow C}$.
\[
    P_{C \leftarrow B} = P_{B \leftarrow C}^{-1} =
    \begin{bmatrix}
        0 & -\frac{1}{2} \\
        1 & \frac{3}{2}
    \end{bmatrix}^{-1}
    =
    \frac{1}{\frac{1}{2}}
    \begin{bmatrix}
        \frac{3}{2} & \frac{1}{2} \\
        -1 & 0
    \end{bmatrix}^{-1}
    = 2
    \begin{bmatrix}
        \frac{3}{2} & \frac{1}{2} \\
        -1 & 0
    \end{bmatrix}^{-1}
\]




\section*{Chapter 5.1}

\subsection*{Problem 2}

Let $A =
\begin{bmatrix}
    -1 & 4 \\
    6 & 9
\end{bmatrix}$
and let $\lambda = -3$.\\
\\[0.1in]
If $\lambda$ is an eigenvalue of $A$ then $A\vec{x} = -3\vec{x}$. This means that $(A + 3I)\vec{x} = \vec{0}$, in other words, the columns of
$\begin{bmatrix}
    2 & 4 \\
    6 & 12
\end{bmatrix}$
are linearly dependent. Since this is obviously true, $\lambda = 3$ is indeed an eigenvalue of $A$.

\subsection*{Problem 4}

Let $A =
\begin{bmatrix}
    5 & 2 \\
    3 & 6
\end{bmatrix}$
and let $\vec{u} =
\begin{bmatrix}
    -1 \\ 1
\end{bmatrix}$. If $\vec{u}$ is an eigenvector of $A$, then $A\vec{u} = c\vec{u}$ where $c \in \mathbb{R}$.
\hfill
$$
A\vec{u} =
\begin{bmatrix}
    5 & 2 \\
    3 & 6
\end{bmatrix}
\begin{bmatrix}
    -1 \\ 1
\end{bmatrix}
=
\begin{bmatrix}
    -5 \\ 3
\end{bmatrix}
+
\begin{bmatrix}
    2 \\ 6
\end{bmatrix}
=
\begin{bmatrix}
    -3 \\ 9
\end{bmatrix}
$$
Because there is no $c \in \mathbb{R}$ such that $c\vec{u} = \begin{bmatrix} -3 \\ 9 \end{bmatrix}$, we know that $\vec{u}$ is not an eigenvector of $A$.

\clearpage
\subsection*{Problem 10}
Let $A =
\begin{bmatrix}
    -4 & 2 \\
    3 & 1
\end{bmatrix}$ and let $\lambda = -5$.
$$
(A+5I) =
\begin{bmatrix}
    1 & 2 \\
    3 & 6
\end{bmatrix}
\backsim
\begin{bmatrix}
    1 & 2 \\
    0 & 0
\end{bmatrix} \Rightarrow
\vec{x} =
\begin{bmatrix}
    x_1 \\ 1
\end{bmatrix}
=
\begin{bmatrix}
    2x_2 \\ x_2
\end{bmatrix}
=
x_2
\begin{bmatrix}
    2 \\ 1
\end{bmatrix}
$$
Let $x_1 = 1$. Then $\left\{\begin{bmatrix} 2 \\ 1 \end{bmatrix}\right\}$ is a basis for the eigenspace of $A$ corresponding to $\lambda$.

\subsection*{Problem 12}

Let $A =
\begin{bmatrix}
    4 & 1 \\
    3 & 6
\end{bmatrix}$, ${\lambda}_1 = 3$, and ${\lambda}_2 = 7$.
$$
(A - 3I) =
\begin{bmatrix}
    1 & 1 \\
    3 & 3
\end{bmatrix}
\backsim
\begin{bmatrix}
    1 & 1 \\
    0 & 0
\end{bmatrix} \Rightarrow
\vec{x} =
\begin{bmatrix}
    x_1 \\ x_2
\end{bmatrix}
=
\begin{bmatrix}
    x_2 \\ x_2
\end{bmatrix}
=
x_2
\begin{bmatrix}
    1 \\ 1
\end{bmatrix}
$$
Let $x_2 = 1$. Then $\left\{\begin{bmatrix} 1 \\ 1 \end{bmatrix}\right\}$ is a basis for the eigenspace of $A$ corresponding to $\lambda_1$.
$$
(A - 3I) =
\begin{bmatrix}
    -3 & 1 \\
    3 & -1
\end{bmatrix}
\backsim
\begin{bmatrix}
    -3 & 1 \\
    0 & 0
\end{bmatrix} \Rightarrow
\vec{x} =
\begin{bmatrix}
    x_1 \\ x_2
\end{bmatrix}
=
\begin{bmatrix}
    \frac{1}{3}x_2 \\ x_2
\end{bmatrix}
=
x_2
\begin{bmatrix}
    \frac{1}{3} \\ 1
\end{bmatrix}
$$
Let $x_2 = 1$. Then $\begin{bmatrix} \frac{1}{3} \\ 1 \end{bmatrix}$ is a basis for the eigenspace of $A$ corresponding to $\lambda_2$

\subsection*{Problem 14}

Let $A =
\begin{bmatrix}
    4 & 0 & -1 \\
    3 & 0 & 3 \\
    2 & -2 & 5
\end{bmatrix}$
and let $\lambda = 3$.
$$
(A - 3I) =
\begin{bmatrix}
    1 & 0 & -1 \\
    3 & -3 & 3 \\
    2 & -2 & 2
\end{bmatrix}
\backsim
\begin{bmatrix}
    1 & 0 & -1 \\
    1 & -1 & 1 \\
    0 & 0 & 0
\end{bmatrix}
\backsim
\begin{bmatrix}
    1 & 0 & -1 \\
    0 & -1 & 2 \\
    0 & 0 & 0
\end{bmatrix}\Rightarrow
\vec{x} =
\begin{bmatrix}
    x_1 \\ x_2 \\ x_3
\end{bmatrix}
=
x_3
\begin{bmatrix}
    -1 \\ -2 \\ 1
\end{bmatrix}
$$
With $x_3 = 1$, $\left\{\begin{bmatrix} -1 \\ -2 \\ 1 \end{bmatrix}\right\}$ forms a basis for the eigenspace of $A$ corresponding to $\lambda$.


\section*{Chapter 5.2}

\subsection*{Problem 2}

Let $A =
\begin{bmatrix}
    -4 & 1 \\
    6 & 1
\end{bmatrix}$. Consider that when $(A - \lambda I)\vec{x} = \vec{0}$, $det(A - \lambda I) = 0$:
$$
det\left(
\begin{bmatrix}
    -4 - \lambda & 1 \\
    6 & 1 - \lambda
\end{bmatrix}
\right) = 0
\Rightarrow
(-4 - \lambda)(1 - \lambda) - 6 = 0
$$
Now, to solve for the eigenvalues of $A$, we need to find the solutions to the equation $(-4 - \lambda)(1 - \lambda) - 6 = 0$:
$$
\lambda^2 + 3\lambda - 10 = 0
\Rightarrow
(\lambda + 5)(\lambda - 2) = 0
$$
Thus, the eigenvalues of $A$ are $\lambda_1 = -5$ and $\lambda_2 = 2$.

\subsection*{Problem 4}

Let $A =
\begin{bmatrix}
    8 & 2 \\
    3 & 3
\end{bmatrix}$. To find the eigenvalues of $A$, we find values such that the equation $det(A - \lambda I) = 0$ is satisfied:
$$
det\left(
\begin{bmatrix}
    8 - \lambda & 2 \\
    3 & 3 - \lambda
\end{bmatrix}
\right) = 0
\Rightarrow
(8 - \lambda)(3 - \lambda) - 6 = 0
$$
Now, we must find the solutions to the equation $\lambda^2 - 11\lambda + 18 = 0$
$$
\lambda^2 - 11\lambda + 18 = 0
\Rightarrow
(\lambda - 9)(\lambda - 2) = 0
$$
Thus, our eigenvalues for $A$ are $\lambda_1 = 9$ and $\lambda_2 = 2$.

\subsection*{Problem 6}

Let $A =
\begin{bmatrix}
    9 & -2 \\
    2 & 5
\end{bmatrix}$. To find the eigenvalues of $A$, we find values such that the equation $det(A - \lambda I) = 0$ is satisfied:
$$
det\left(
\begin{bmatrix}
    9 - \lambda & -2 \\
    2 & 5 - \lambda
\end{bmatrix}
\right) = 0
\Rightarrow
(9 - \lambda)(5 - \lambda) + 4 = 0
$$
Now we must solve for equation $\lambda^2 + 49 - 14\lambda = 0$:
$$
\lambda^2 + 49 - 14\lambda = 0
\Rightarrow
(\lambda - 7)(\lambda - 7) = 0
$$
This means that $A$ only has a single eigenvalue $\lambda = 7$


\end{document}

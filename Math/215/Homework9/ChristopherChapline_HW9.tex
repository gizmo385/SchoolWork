\documentclass{article}%
\usepackage{amsmath}
\usepackage{amsfonts}
\usepackage{amssymb}
\usepackage{graphicx}
\usepackage{tikz}
\usepackage{hyperref}%
\setcounter{MaxMatrixCols}{30}
%TCIDATA{OutputFilter=late$x2$.dll}
%TCIDATA{Version=5.00.0.2552}
%TCIDATA{CSTFile=40 LaTeX article.cst}
%TCIDATA{Created=Thursday, August 21, 2008 14:03:59}
%TCIDATA{LastRevised=Wednesday, October 01, 2014 12:46:33}
%TCIDATA{<META NAME="GraphicsSave" CONTENT="32">}
%TCIDATA{<META NAME="SaveForMode" CONTENT="1">}
%TCIDATA{<META NAME="DocumentShell" CONTENT="Standard LaTeX\Blank - Standard LaTeX Article">}
%TCIDATA{Language=American English}
\newtheorem{theorem}{Theorem}
\newtheorem{acknowledgement}[theorem]{Acknowledgement}
\newtheorem{algorithm}[theorem]{Algorithm}
\newtheorem{axiom}[theorem]{Axiom}
\newtheorem{case}[theorem]{Case}
\newtheorem{claim}[theorem]{Claim}
\newtheorem{conclusion}[theorem]{Conclusion}
\newtheorem{condition}[theorem]{Condition}
\newtheorem{conjecture}[theorem]{Conjecture}
\newtheorem{corollary}[theorem]{Corollary}
\newtheorem{criterion}[theorem]{Criterion}
\newtheorem{definition}[theorem]{Definition}
\newtheorem{example}[theorem]{Example}
\newtheorem{exercise}[theorem]{Exercise}
\newtheorem{lemma}[theorem]{Lemma}
\newtheorem{notation}[theorem]{Notation}
\newtheorem{problem}[theorem]{Problem}
\newtheorem{proposition}[theorem]{Proposition}
\newtheorem{remark}[theorem]{Remark}
\newtheorem{solution}[theorem]{Solution}
\newtheorem{summary}[theorem]{Summary}
\newenvironment{proof}[1][Proof]{\noindent\textbf{#1.} }{\ \rule{0.5em}{0.5em}}

\usepackage{fancyhdr}
\setlength\headheight{26pt}
\pagestyle{fancy}
\lhead{{\footnotesize Homework 9}}
\rhead{{\footnotesize Christopher Chapline}}
\begin{document}

\section*{Chapter 4.1}

\subsection*{Problem 6}

The set described is not a subspace of $\mathbb{P}_n$ because it not closed under scalar multiplication. Take an arbitrary element from the set, $p(t) = t^2 + a$ and multiply it by some
$b \in \mathbb{R}$. $b \cdot p(t) = b(t^2 + a) = bt^2 + ba$. Because $bt^2 + ba$ is not in the set, this is not a subspace.

\subsection*{Problem 8}

This set is a subspace of $\mathbb{P}_n$. Because the zero polynomial satisfies $p(0) = 0$, the zero polynomial is in this set. If $p_1(t) = 0$ and $p_2(t) = 0$, then $p_1_ + p_2 = 0$.
Also, if $p_1(t) = 0$, then $p_1(t) \cdot a = 0$ when $a \in \mathbb{R}$.\\
\\
Because the set contains the zero polynomial and is closed under vector addition and scalar multiplication, it is a subspace of $\mathbb{P}_n$.

\subsection*{Problem 11}

Let $W = \left\{
    \begin{bmatrix}
        2b + 3c \\
        -b \\
        2c
    \end{bmatrix}
    : b, c \in \mathbb{R}
\right\}$.
We can represent the vector that describes the elements of $W$ as the following linear combination:\\
\\
$
\begin{bmatrix}
    2b + 3c \\
    -b \\
    2c
\end{bmatrix}
=
b
\begin{bmatrix}
    2 \\ - 1 \\ 0
\end{bmatrix}
+
c
\begin{bmatrix}
    3 \\ 0 \\ 2
\end{bmatrix}
$\\
\\
Thus, we can say
$
W = Span
\left\{
    \begin{bmatrix}
        2 \\ - 1 \\ 0
    \end{bmatrix}
    ,
    \begin{bmatrix}
        3 \\ 0 \\ 2
    \end{bmatrix}
\right\}
$. Because we can represent $W$ as the span of some vectors, we know that $W$ is a subspace of $\mathbb{R}^3$ by \textbf{theorem 1}.

\subsection*{Problem 12}

We can the vector that describes the elements of $W$ as the following linear combination:\\
\\
$
\begin{bmatrix}
    2s + 4t \\
    2s \\
    2s - 3t \\
    5t
\end{bmatrix}
=
\begin{bmatrix}
    1 \\ 2 \\ 2 \\ 0
\end{bmatrix}
+
\begin{bmatrix}
    4 \\ 2 \\ -3 \\ 5
\end{bmatrix}
$\\
\\
Because we can represent $W$ as the span of two vectors in $\mathbb{R}^4$, we know that $W$ is a subspace of $\mathbb{R}^4$.

\subsection*{Problem 16}

Take two arbitrary vectors of this form and add them together:\\
\\
$
\begin{bmatrix}
    1 \\
    3a_1 - 5b_1 \\
    3b_1 + 2a_1
\end{bmatrix}
+
\begin{bmatrix}
    1 \\
    3a_2 - 5b_2 \\
    3b_2 + 2a_2
\end{bmatrix}
=
\begin{bmatrix}
    2 \\
    3(a_1 + a_2) - 5(b_1 + b_2) \\
    2(a_1 + a_2) + 3(b_1 + b_2)
\end{bmatrix}
$\\
\\
Because the first row is not of the form
$\begin{bmatrix}
    1 \\
    3a - 5b \\
    3b + 2a
\end{bmatrix}$
, the vectors of this form are not closed under vector addition and thus do not form a vector subspace.

\subsection*{Problem 18}
We can represent the vector that forms the elements of this set as a linear combination:
$
\begin{bmatrix}
    4a + 3b \\
    0 \\
    a + 3b + c \\
    3b - 2c
\end{bmatrix}
=
a
\begin{bmatrix}
    3 \\ 0 \\ 1 \\ 0
\end{bmatrix}
+
b
\begin{bmatrix}
    3 \\ 0 \\ 3 \\ 3
\end{bmatrix}
+
c
\begin{bmatrix}
    0 \\ 0 \\ 1 \\ -2
\end{bmatrix}$\\
\\
This means that we can define the set of vectors of that form to be
$
Span
\left\{
    \begin{bmatrix}
        3 \\ 0 \\ 1 \\ 0
    \end{bmatrix}
    ,
    \begin{bmatrix}
        3 \\ 0 \\ 3 \\ 3
    \end{bmatrix}
    ,
    \begin{bmatrix}
        0 \\ 0 \\ 1 \\ -2
    \end{bmatrix}
\right\}
$.\\
\\
Because we can define vectors of this form as a span, we know that those vectors form a subspace of $\mathbb{R}^4$.

\subsection*{Problem 21}

First, we identify that the matrix of all zeroes is in $H$ when $a = 0, b = 0$, and $c = 0$.\\
\\
Second, we will take two arbitrary elements of $H$ and add them:\\
\\
$
\begin{bmatrix}
    a_1 & b_1 \\
    0 & c_1
\end{bmatrix}
+
\begin{bmatrix}
    a_2 & b_2 \\
    0 & c_2
\end{bmatrix}
=
\begin{bmatrix}
    a_1 + a_2 & b_1 + b_2 \\
    0 & c_1 + c_2
\end{bmatrix}
$\\
\\
Because the result of this addition is in $H$, the set is closed under addition. Lastly, we will check that $H$ is closed under scalar multiplication:\\
\\
$
k
\begin{bmatrix}
    a & b \\
    0 & c
\end{bmatrix}
=
\begin{bmatrix}
    ka & kb \\
    0 & kc
\end{bmatrix}
$\\
\\
This result is in $H$, meaning that $H$ is closed under scalar multiplication. Because all 3 of these properties hold, $H$ is a vector subspace.


\section*{Chapter 4.2}

\subsection*{Problem 2}

Let
$\vec{w} =
\begin{bmatrix}
    1 \\ -1 \\ 1
\end{bmatrix}, A =
\begin{bmatrix}
    2 & 6 & 4 \\
    -3 & 2 & 5 \\
    -5 & -4 & 1
\end{bmatrix}$.\\
\\
If $\vec{w} \in Null(A)$, then $A\vec{w} = \vec{0}$. We will calculate $A\vec{w}$:\\
\\
$
\begin{bmatrix}
    2 & 6 & 4 \\
    -3 & 2 & 5 \\
    -5 & -4 & 1
\end{bmatrix}
\begin{bmatrix}
    1 \\ -1 \\ 1
\end{bmatrix}
=
\begin{bmatrix}
    2 \\ -3 \\ -5
\end{bmatrix}
+
\begin{bmatrix}
    -6 \\ -2 \\ 4
\end{bmatrix}
+
\begin{bmatrix}
    4 \\ 5 \\ 1
\end{bmatrix}
=
\begin{bmatrix}
    0 \\ 0 \\ 0
\end{bmatrix}$\\
\\
Because $A\vec{w} = \vec{0}$, we know that $\vec{w} \in Null(A)$.

\subsection*{Problem 4}

Let $A =
\begin{bmatrix}
    1 & -3 & 2 & 0 \\
    0 & 0 & 3 & 0
\end{bmatrix}$. \\
\\
To find a description of the $Null(A)$, we will solve the homogenous equation, $A\vec{x} = \vec{0}$:\\
\\
$
[A | \vec{0}] \backsim
\begin{bmatrix}
    1 & -3 & 2 & 0 & \vline & 0\\
    0 & 0 & 3 & 0 & \vline & 0
\end{bmatrix}
\backsim
\begin{bmatrix}
    1 & -3 & 2 & 0 & \vline & 0\\
    0 & 0 & 1 & 0 & \vline & 0
\end{bmatrix}
\backsim
\begin{bmatrix}
    1 & -3 & 0 & 0 & \vline & 0\\
    0 & 0 & 1 & 0 & \vline & 0
\end{bmatrix}
$\\
\\[0.1in]
We can rewrite this like so:\\
\\
$
\vec{x} =
\begin{bmatrix}
    x_1 \\ x_2 \\ x_3 \\ x_4
\end{bmatrix}
=
\begin{bmatrix}
    3 x_2 \\
    x_2 \\
    x_3 \\
    x_4
\end{bmatrix}
=
\begin{bmatrix}
    3 x_2 \\
    x_2 \\
    0 \\
    x_4
\end{bmatrix}
=
x_2
\begin{bmatrix}
    3 \\
    1 \\
    0 \\
    0
\end{bmatrix}
+
x_4
\begin{bmatrix}
    0 \\
    0 \\
    0 \\
    1
\end{bmatrix}
: x_2, x_4 \in \mathbb{R}
$\\
\\[0.07in]
This means that we can describe $Null(A)$ as
$Span
\left\{
    \begin{bmatrix}
        3 \\
        1 \\
        0 \\
        0
    \end{bmatrix}
    ,
    \begin{bmatrix}
        0 \\
        0 \\
        0 \\
        1
    \end{bmatrix}
\right\}
$

\subsection*{Problem 6}

Let $A =
\begin{bmatrix}
    1 & 3 & -4 & -3 & 1 \\
    0 & 1 & -3 & 1 & 0 \\
    0 & 0 & 0 & 0 & 0
\end{bmatrix}
$.\\
\\
To find an explicit description of $Null(A)$, we will solve the homogenous equation $A\vec{x} = \vec{0}$:\\
\\
$
[A | \vec{0}] =
\begin{bmatrix}
    1 & 3 & -4 & -3 & 1 & \vline & 0 \\
    0 & 1 & -3 & 1 & 0 & \vline & 0 \\
    0 & 0 & 0 & 0 & 0 & \vline & 0
\end{bmatrix}
\backsim
\begin{bmatrix}
    1 & 0 & 5 & -6 & 1 & \vline & 0 \\
    0 & 1 & -3 & 1 & 0 & \vline & 0 \\
    0 & 0 & 0 & 0 & 0 & \vline & 0
\end{bmatrix}
$\\
\\
We an rewrite this like so:\\
\\
$\vec{x} =
\begin{bmatrix}
    x_1 \\ x_2 \\ x_3 \\ x_4 \\ x_5
\end{bmatrix}
=
\begin{bmatrix}
    -5x_3 + 6x_4 - x_5\\
    3x_3 - x_4
    x_3
    x_4
    x_5
\end{bmatrix}
=
x_3
\begin{bmatrix}
    -5 \\ 3 \\ 1 \\ 0 \\ 0
\end{bmatrix}
x_4
\begin{bmatrix}
    6 \\ -1 \\ 0 \\ 1 \\ 0
\end{bmatrix}
x_5
\begin{bmatrix}
    -1 \\ 0 \\ 0 \\ 0 \\ 1
\end{bmatrix}
$\\
\\[0.1in]
This means that we can describe $Null(A)$ as
$Span
\left\{
    \begin{bmatrix}
        -5 \\ 3 \\ 1 \\ 0 \\ 0
    \end{bmatrix}
    ,
    \begin{bmatrix}
        6 \\ -1 \\ 0 \\ 1 \\ 0
    \end{bmatrix}
    ,
    \begin{bmatrix}
        -1 \\ 0 \\ 0 \\ 0 \\ 1
    \end{bmatrix}
\right\}


\subsection*{Problem 16}

Let
$H =
\left\{
    \begin{bmatrix}
        b - c \\
        2b + 3d \\
        b + 3c - 3d \\
        c + d
    \end{bmatrix}
    : b, c, d \in \mathbb{R}
\right\}$.\\
\\
We can rewrite the vector describing elements in this set as the following linear combination:
\\[0.05in]
$
H =
b
\begin{bmatrix}
    1 \\ 2 \\ 1 \\ 0
\end{bmatrix}
+
c
\begin{bmatrix}
    -1 \\ 0 \\ 3 \\ 1
\end{bmatrix}
+
d
\begin{bmatrix}
    0 \\ 3 \\ -3 \\ 1
\end{bmatrix}
$\\
\\[0.1in]
We can conclude that:
$A =
\begin{bmatrix}
    1 & -1 & 0 \\
    2 & 0 & 3 \\
    1 & 3 & -3 \\
    0 & 1 & 1
\end{bmatrix}
$

\subsection*{Problem 24}

Let
$A =
\begin{bmatrix}
    10 & -8 & -2 & -2 \\
    0 & 2 & 2 & -2 \\
    1 & -1 & 6 & 0 \\
    1 & 1 & 0 & -2
\end{bmatrix}
, w =
\begin{bmatrix}
    2 \\ 2 \\ 0 \\ 2
\end{bmatrix}$.\\
\\
To determine if $w \in Null(A)$, we can check to see if $A\vec{w} = \vec{0}$:\\
\\
$
\begin{bmatrix}
    10 & -8 & -2 & -2 \\
    0 & 2 & 2 & -2 \\
    1 & -1 & 6 & 0 \\
    1 & 1 & 0 & -2
\end{bmatrix}
\begin{bmatrix}
    2 \\ 2 \\ 0 \\ 2
\end{bmatrix}
=
\begin{bmatrix}
    20 \\ 0 \\ 2 \\ 2
\end{bmatrix}
+
\begin{bmatrix}
    -16 \\ 4 \\ -2 \\ 2
\end{bmatrix}
+
\begin{bmatrix}
    0 \\ 0 \\ 0 \\ 0
\end{bmatrix}
+
\begin{bmatrix}
    -4 \\ -4 \\ 0 \\ -4
\end{bmatrix}
=
\begin{bmatrix}
    0 \\ 0 \\ 0 \\ 0
\end{bmatrix}
$\\
\\[0.1in]
Because $A\vec{w} = \vec{0}$, we can conclude that $w \in Null(A)$.\\
\\[0.15in]
Now, to determine if $\vec{w} \in Coll(A)$, we must show whether or not $[A | \vec{w}]$ is consistent:\\
\\
$
[A | \vec{w}] \backsim
\begin{bmatrix}
    10 & -8 & -2 & -2 & \vline & 2 \\
    0 & 2 & 2 & -2 & \vline & 2 \\
    1 & -1 & 6 & 0 & \vline & 0 \\
    1 & 1 & 0 & -2 & \vline & 2
\end{bmatrix}
\backsim
\begin{bmatrix}
    1 & 0 & 0 & -1 & 1 \\
    0 & 1 & 0 & -1 & 1 \\
    0 & 0 & 1 & 0 & 0 \\
    0 & 0 & 0 & 0 & 0
\end{bmatrix}
$\\
\\
Because this augmented matrix is consistent, we know that $\vec{w} \in Coll(A)$.

\end{document}

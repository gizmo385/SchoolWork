\documentclass{article}%
\usepackage{amsmath}
\usepackage{amsfonts}
\usepackage{amssymb}
\usepackage{graphicx}
\usepackage{tikz}
\usepackage{hyperref}%
\setcounter{MaxMatrixCols}{30}
%TCIDATA{OutputFilter=late$x2$.dll}
%TCIDATA{Version=5.00.0.2552}
%TCIDATA{CSTFile=40 LaTeX article.cst}
%TCIDATA{Created=Thursday, August 21, 2008 14:03:59}
%TCIDATA{LastRevised=Wednesday, October 01, 2014 12:46:33}
%TCIDATA{<META NAME="GraphicsSave" CONTENT="32">}
%TCIDATA{<META NAME="SaveForMode" CONTENT="1">}
%TCIDATA{<META NAME="DocumentShell" CONTENT="Standard LaTeX\Blank - Standard LaTeX Article">}
%TCIDATA{Language=American English}
\newtheorem{theorem}{Theorem}
\newtheorem{acknowledgement}[theorem]{Acknowledgement}
\newtheorem{algorithm}[theorem]{Algorithm}
\newtheorem{axiom}[theorem]{Axiom}
\newtheorem{case}[theorem]{Case}
\newtheorem{claim}[theorem]{Claim}
\newtheorem{conclusion}[theorem]{Conclusion}
\newtheorem{condition}[theorem]{Condition}
\newtheorem{conjecture}[theorem]{Conjecture}
\newtheorem{corollary}[theorem]{Corollary}
\newtheorem{criterion}[theorem]{Criterion}
\newtheorem{definition}[theorem]{Definition}
\newtheorem{example}[theorem]{Example}
\newtheorem{exercise}[theorem]{Exercise}
\newtheorem{lemma}[theorem]{Lemma}
\newtheorem{notation}[theorem]{Notation}
\newtheorem{problem}[theorem]{Problem}
\newtheorem{proposition}[theorem]{Proposition}
\newtheorem{remark}[theorem]{Remark}
\newtheorem{solution}[theorem]{Solution}
\newtheorem{summary}[theorem]{Summary}
\newenvironment{proof}[1][Proof]{\noindent\textbf{#1.} }{\ \rule{0.5em}{0.5em}}
\usepackage{fancyhdr}
\setlength\headheight{26pt}
\pagestyle{fancy}
\lhead{{\footnotesize Math 215 - Homework 8}}
\rhead{{\footnotesize Christopher Chapline}}
\begin{document}

\section*{Chapter 2.2}

\subsubsection*{Problem 3}

Let $A = $
\begin{bmatrix}
    7 & 3 \\
    -6 & -3
\end{bmatrix}.
The inverse of a 2x2 matrix is:
$\frac{1}{ad-bc}$
\begin{bmatrix}
    a & b \\
    c & d
\end{bmatrix}.
Thus, $A^{-1} = \frac{1}{(7)(-3) - (3)(-6)}$
\begin{bmatrix}
    -3 & -3 \\
    6 & 7
\end{bmatrix}
$= \frac{1}{3}$
\begin{bmatrix}
    -3 & -3 \\
    6 & 7
\end{bmatrix}.

\subsubsection*{Problem 6}
Solve the linear system:
\begin{cases}
    $7x_1 + 3x_2 = 9$\\
    $-6x_1 - 3x_2 = ?$
\end{cases}\\[0.1in]

$A\vec{x} =$
\begin{bmatrix}
    9 \\ ?
\end{bmatrix}
$\Rightarrow \vec{x} = A^{-1}$
\begin{bmatrix}
    9 \\ ?
\end{bmatrix}
$\Rightarrow \vec{x} = \frac{1}{3}$
\begin{bmatrix}
    -3 & -3 \\
    6 & 7
\end{bmatrix}
\begin{bmatrix}
    9 \\ ?
\end{bmatrix}

\subsubsection*{Problem 8}
$A = PBP^{-1} \Rightarrow P^{-1}A = BP^{-1} \Rightarrow P^{-1}AP = B$.

\subsubsection*{Problem 32}

Find $A^{-1}$ if it exists when $A =$
\begin{bmatrix}
    1 & 2 & -1 \\
    -4 & -7 & 3 \\
    -2 & -6 & 4
\end{bmatrix}.
We will augment by $I_3$ and row reduce.\\
\\[0.1in]
\begin{bmatrix}
    1 & 2 & -1 & \vline & 1 & 0 & 0 \\
    -4 & -7 & 3 & \vline & 0 & 1 & 0 \\
    -2 & -6 & 4 & \vline & 0 & 0 & 1
\end{bmatrix}
$\backsim$
\begin{bmatrix}
    1 & 2 & -1 & \vline & 1 & 0 & 0 \\
    0 & 1 & -1 & \vline & 4 & 1 & 0 \\
    0 & -2 & 2 & \vline & 2 & 0 & 1
\end{bmatrix}
$\backsim$
\begin{bmatrix}
    1 & 2 & -1 & \vline & 1 & 0 & 0 \\
    0 & 1 & -1 & \vline & 4 & 1 & 0 \\
    0 & 0 & 0 & \vline & 10 & 2 & 1
\end{bmatrix}\\
\\[0.1in]
Because $A$ is not row equivalent with $I_3$, $A$ is not invertible.

\section*{Chapter 2.3}

\subsubsection*{Problem 12}
\paragraph{(a)} False. Matrix multiplication is not commutative.
\paragraph{(b)} False. If the map $T$ were onto or one-to-one, this would be true.
\paragraph{(c)} Flase. Not every $nxn$ matrix spans $\mathbb{R}^n$
\paragraph{(d)} True. If $A$ had $n$ pivots, $A\vec{x} = \vec{b}$ would only have the trivial solution by the Invertible Matrix Theorem.
\paragraph{(e)} True. If a matrix is invertible, so is its transpose. This is Theorem 6.

\subsubsection*{Problem 20}
Yes, it is possible. According to the invertible matrix theorem, if an $nxn$ matrix $A$ spans $\mathbb{R}^n$, then $A\vec{x} = \vec{b}$ has \textbf{at least one} solution for all $\vec{b}$ in $\mathbb{R}^n$.

\subsubsection*{Problem 21}
The existence of some $\vec{v}$ such that $C\vec{u} = \vec{v}$ has more than one solution does not prevent $C$ from spanning $\mathbb{R}^n$. As in number 20, the invertible matrix theorem says that if $C\vec{u} = \vec{v}$ has \textbf{at least one} solution for all $\vec{v}$ in $\mathbb{R}^n$, then it spans $\mathbb{R}^n$.

\subsubsection*{Problem 28}
Theorem 6 states that $(AB)^{-1} = B^{-1}A^{-1}$. This equality shows that for $AB$ to be invertible, it must be representable as the product of 2 invertible matrices.

\section{Chapter 3.1}

\subsubsection{Problem 10}
Calculate the determinant of $A$ when $A = $
\begin{bmatrix}
    1 & -2 & 5 & 2 \\
    0 & 0 & 3 & 0 \\
    2 & -6 & -7 & 0 \\
    5 & 0 & 4 & 4
\end{bmatrix}.\\

\begin{array}{|l l l l|}
    1 & -2 & 5 & 2 \\
    0 & 0 & 3 & 0 \\
    2 & -6 & -7 & 0 \\
    5 & 0 & 4 & 4
$\end{array} = 3 * $
\begin{array}{|l l l|}
    1 & -2 & 2 \\
    2 & -6 & 5 \\
    5 & 0 & 4
$\end{array} =$
$-15 *$
\begin{array}{|l l|}
    -2 & -2 \\
    -6 & 5
$\end{array}$
$ -12 *$
\begin{array}{|l l|}
    1 & -2 \\
    2 & -6
$\end{array}$
\\[0.1in]
$det(A) = -15(-10 + 12) - 12(-6 + 4) = -30 + 24 = -6$

\subsubsection{Problem 12}
\begin{array}{|l l l l|}
    1 & -2 & 5 & 2 \\
    0 & 0 & 3 & 0 \\
    2 & -6 & -7 & 0 \\
    5 & 0 & 4 & 4
$\end{array} = 4*$
\begin{array}{|l l l|}
    -1 & 0 & 0 \\
    6 & 3 & -3 \\
    -8 & 4 & -3
$\end{array} = -4*$
\begin{array}{|l l|}
    3 & -3 \\
    4 & -3
$\end{array} = -4(-9 + 12) = -4 * 3 = -12$.

\clearpage
\subsubsection{Problem 14}
\begin{array}{|l l l l l|}
    6 & 3 & 2 & 4 & 0 \\
    9 & 0 & -4 & 1 & 0 \\
    8 & -5 & 6 & 7 & 1 \\
    3 & 0 & 0 & 0 & 0 \\
    4 & 2 & 3 & 2 & 0
$\end{array} = $
\begin{array}{|l l l l|}
    6 & 3 & 2 & 4 \\
    9 & 0 & -4 & 1 \\
    3 & 0 & 0 & 0 \\
    4 & 2 & 3 & 2
$\end{array} = 3 *$
\begin{array}{|l l l|}
    3 & 2 & 4 \\
    0 & -4 & 1 \\
    2 & 3 & 2
$\end{array}$
\\[0.1in]
$= 12 *$
\begin{array}{|l l|}
    3 & 4 \\
    2 & 2
$\end{array} - 3 *$
\begin{array}{|l l|}
    3 & 2 \\
    2 & 3
$\end{array} = 12(6-8) - 3(9-4) = -24 - 15 = -39$


\end{document}

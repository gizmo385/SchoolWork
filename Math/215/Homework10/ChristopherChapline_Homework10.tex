\documentclass{article}%
\usepackage{amsmath}
\usepackage{amsfonts}
\usepackage{amssymb}
\usepackage{graphicx}
\usepackage{tikz}
\usepackage{hyperref}%
\setcounter{MaxMatrixCols}{30}
%TCIDATA{OutputFilter=late$x2$.dll}
%TCIDATA{Version=5.00.0.2552}
%TCIDATA{CSTFile=40 LaTeX article.cst}
%TCIDATA{Created=Thursday, August 21, 2008 14:03:59}
%TCIDATA{LastRevised=Wednesday, October 01, 2014 12:46:33}
%TCIDATA{<META NAME="GraphicsSave" CONTENT="32">}
%TCIDATA{<META NAME="SaveForMode" CONTENT="1">}
%TCIDATA{<META NAME="DocumentShell" CONTENT="Standard LaTeX\Blank - Standard LaTeX Article">}
%TCIDATA{Language=American English}
\newtheorem{theorem}{Theorem}
\newtheorem{acknowledgement}[theorem]{Acknowledgement}
\newtheorem{algorithm}[theorem]{Algorithm}
\newtheorem{axiom}[theorem]{Axiom}
\newtheorem{case}[theorem]{Case}
\newtheorem{claim}[theorem]{Claim}
\newtheorem{conclusion}[theorem]{Conclusion}
\newtheorem{condition}[theorem]{Condition}
\newtheorem{conjecture}[theorem]{Conjecture}
\newtheorem{corollary}[theorem]{Corollary}
\newtheorem{criterion}[theorem]{Criterion}
\newtheorem{definition}[theorem]{Definition}
\newtheorem{example}[theorem]{Example}
\newtheorem{exercise}[theorem]{Exercise}
\newtheorem{lemma}[theorem]{Lemma}
\newtheorem{notation}[theorem]{Notation}
\newtheorem{problem}[theorem]{Problem}
\newtheorem{proposition}[theorem]{Proposition}
\newtheorem{remark}[theorem]{Remark}
\newtheorem{solution}[theorem]{Solution}
\newtheorem{summary}[theorem]{Summary}
\newenvironment{proof}[1][Proof]{\noindent\textbf{#1.} }{\ \rule{0.5em}{0.5em}}

\usepackage{fancyhdr}
\setlength\headheight{26pt}
\pagestyle{fancy}
\lhead{{\footnotesize Math 215 - Homework 10}}
\rhead{{\footnotesize Christopher Chapline}}
\begin{document}

\section*{Section 4.3}

\subsection*{Problem 14}

Let $A = $
\begin{bmatrix}
    1 & 2 & 3 & -4 & 8 \\
    1 & 2 & 0 & 2 & 8 \\
    2 & 4 & -3 & 10 & 9 \\
    3 & 6 & 0 & 6 & 9
\end{bmatrix},
$B = $
\begin{bmatrix}
    1 & 2 & 0 & 2 & 5 \\
    0 & 0 & 3 & -6 & 3 \\
    0 & 0 & 0 & 0 & -7 \\
    0 & 0 & 0 & 0 & 0
\end{bmatrix}\\
\\[0.1in]
To find $Coll(A)$, we will identify the pivot columns of $A$. These columns are
$\left\{
    \begin{bmatrix}
        1 \\ 1 \\ 2 \\ 3
    \end{bmatrix},
    \begin{bmatrix}
        3 \\ 0 \\ -3 \\ 0
    \end{bmatrix},
    \begin{bmatrix}
        8 \\ 8 \\ 9 \\ 9
    \end{bmatrix}
\right\}$.
Thus, we can say that $Coll(A) = Span
\left\{
    \begin{bmatrix}
        1 \\ 1 \\ 2 \\ 3
    \end{bmatrix},
    \begin{bmatrix}
        3 \\ 0 \\ -3 \\ 0
    \end{bmatrix},
    \begin{bmatrix}
        8 \\ 8 \\ 9 \\ 9
    \end{bmatrix}
\right\}$.\\
\\[0.1in]
To find $Null(A)$, we will determine when $A\vec{x} = \vec{0}$\\
\\
$\vec{x} =
\begin{bmatrix}
    x_1 \\ x_2 \\ x_3 \\ x_4 \\ x_5
\end{bmatrix}=
\begin{bmatrix}
    -2x_2 - 2x_4 - 5x_5 \\
    x_2 \\
    2x_4 - x_5 \\
    x_4 \\
    x_5
\end{bmatrix} =
x_2
\begin{bmatrix}
    -2 \\
    1 \\
    0 \\
    0 \\
    0
\end{bmatrix}
+
x_4
\begin{bmatrix}
    -2 \\
    0 \\
    2 \\
    1 \\
    0
\end{bmatrix}
+
x_5
\begin{bmatrix}
    -5 \\
    0 \\
    -1 \\
    0 \\
    1
\end{bmatrix}$\\
\\[0.1in]
Thus, $Null(A) =
\left\{
\begin{bmatrix}
    -2 \\
    1 \\
    0 \\
    0 \\
    0
\end{bmatrix},
\begin{bmatrix}
    -2 \\
    0 \\
    2 \\
    1 \\
    0
\end{bmatrix},
\begin{bmatrix}
    -5 \\
    0 \\
    -1 \\
    0 \\
    1
\end{bmatrix}
\right\}$

\subsection*{Problem 16}

Let $V =
\left\{
    \begin{bmatrix}
        1 \\ 0 \\ 0 \\ 1
    \end{bmatrix},
    \begin{bmatrix}
        -2 \\ 0 \\ 0 \\ 2
    \end{bmatrix},
    \begin{bmatrix}
        3 \\ -1 \\ 1 \\ -1
    \end{bmatrix},
    \begin{bmatrix}
        5 \\ -3 \\ 3 \\ -4
    \end{bmatrix},
    \begin{bmatrix}
        2 \\ -1 \\ 1 \\ 0
    \end{bmatrix}
\right\}$. To find a basis for the span of the vectors in $V$, we need to find the linearly independent vectors of $V$.\\
\\[0.05in]
Let $A =
\begin{bmatrix}
    1 & -2 & 3 & 5 & 2 \\
    0 & 0 & -1 & -3 & -1 \\
    0 & 0 & 1 & 3 & 1 \\
    1 & 2 & -1 & -4 & 0
\end{bmatrix}
\backsim
\begin{bmatrix}
    1 & -2 & 3 & 5 & 2 \\
    0 & 1 & -1 & -9 & -2 \\
    0 & 0 & 1 & 3 & 1 \\
    0 & 0 & 0 & 0 & 0
\end{bmatrix}$
\clearpage
\noindent Taking only the pivot columns, a basis for the space spanned by $V$ is
$\left\{
    \begin{bmatrix}
        1 \\ 0 \\ 0 \\ 0
    \end{bmatrix},
    \begin{bmatrix}
        -2 \\ 1 \\ 0 \\ 0
    \end{bmatrix},
    \begin{bmatrix}
        3 \\ -1 \\ 1 \\ 0
    \end{bmatrix}
\right\}
$

\subsection*{Problem 24}

The matrix whose columns are the vectors of \mathcal{B} must have a pivot in every column because the set is linearly independent. Because each vector
exists in $\mathbb{R}^n$, the matrix is square. Since the matrix is square and its columns are linearly independent, we know that the vectors span
$\mathbb{R}^n$ by the Invertible Matrix Theorem. Since the vectors form a linearly independent spanning set, they form a basis of $\mathbb{R}^n$.

\section*{Section 4.4}

\subsection*{Problem 2}

Let $\mathcal{B} =
\left\{
    \begin{bmatrix}
        3 \\ -5
    \end{bmatrix},
    \begin{bmatrix}
        -4 \\ 6
    \end{bmatrix}
\right\}$
and let $[\vec{x}]_\mathcal{B} =
\begin{bmatrix}
    -2 \\ 5
\end{bmatrix}$.\\
\\[0.05in]
$\vec{x} = (-2)
\begin{bmatrix}
    3 \\ -5
\end{bmatrix}
+
5
\begin{bmatrix}
    -4 \\ 6
\end{bmatrix}
=
\begin{bmatrix}
    -6 \\ 10
\end{bmatrix}
+
\begin{bmatrix}
    -20 \\ 30
\end{bmatrix}
=
\begin{bmatrix}
    -26 \\ 40
\end{bmatrix}$

\subsection*{Problem 4}

Let $\mathcal{B} =
\left\{
    \begin{bmatrix}
        -2 \\ 2 \\ 0
    \end{bmatrix},
    \begin{bmatrix}
        3 \\ 0 \\ 2
    \end{bmatrix},
    \begin{bmatrix}
        4 \\ -1 \\ 3
    \end{bmatrix}
\right\}$
and let $[\vec{x}]_\mathcal{B} =
\begin{bmatrix}
    1 \\ 0 \\ -2
\end{bmatrix}$.\\
\\[0.05in]
$\vec{x} =
\begin{bmatrix}
    -2 \\ 2 \\ 0
\end{bmatrix}
+
\begin{bmatrix}
    -8 \\ 2 \\ -6
\end{bmatrix}
=
\begin{bmatrix}
    -10 \\ 4 \\ -6
\end{bmatrix}$

\subsection*{Problem 13}

Let $\mathcal{B} = \{1 + t^2, t + t^2, 1 + 2t + t^2\}$ be a basis for $\mathbb{P}_2$ and let $p(t) = 1 + 4t + 7t^2$. First, we will
write $p(t)$ in terms of the basis $\mathbb{C} = \{1, t, t^2\}$: $[p(t)]_\mathbb{C} =
\begin{bmatrix}
    1 \\ 4 \\ 7
\end{bmatrix}$.\\
\\[0.05in]
To rewrite this relative to $\mathbb{C}$, we will augment the vectors in $\mathcal{B}$ relative to the standard basis and find a solution
to the system:\\
\\
$[1 + t^2]_\mathbb{C} =
\begin{bmatrix}
    1 \\ 0 \\ 1
\end{bmatrix}$
\hfill
$[t + t^2]_\mathbb{C} =
\begin{bmatrix}
    1 \\ 1 \\ 0
\end{bmatrix}$
\hfill
$[1 + 2t + t^2]_\mathbb{C} =
\begin{bmatrix}
    1 \\ 2 \\ 1
\end{bmatrix}$\\
\\[0.7in]
$
\begin{bmatrix}
    1 & 1 & 1 & \vline & 1 \\
    0 & 1 & 2 & \vline & 4 \\
    1 & 0 & 1 & \vline & 7
\end{bmatrix}
\backsim
\begin{bmatrix}
    0 & 1 & 0 & \vline & 8 \\
    0 & 1 & 2 & \vline & 4 \\
    1 & 0 & 1 & \vline & 7
\end{bmatrix}
\backsim
\begin{bmatrix}
    0 & 1 & 0 & \vline & 8 \\
    0 & 0 & 1 & \vline & -2 \\
    1 & 0 & 1 & \vline & 7
\end{bmatrix}
\backsim\\
\\[0.07in]
\begin{bmatrix}
    0 & 1 & 0 & \vline & 8 \\
    0 & 0 & 1 & \vline & -2 \\
    1 & 0 & 0 & \vline & 9
\end{bmatrix}
\backsim
\begin{bmatrix}
    1 & 0 & 0 & \vline & 9 \\
    0 & 1 & 0 & \vline & 8 \\
    0 & 0 & 1 & \vline & -2
\end{bmatrix}
$\\
\\[0.1in]
Thus, $[p(t)]_\mathcal{B} =
\begin{bmatrix}
    9 \\ 8 \\ -2
\end{bmatrix}$.



\subsection*{Problem 28}

We will use the basis $\mathcal{B} = \{t^3, t^2, t, 1\}$ for $\mathbb{P}_3}$. Rewriting each polynomial as a column vector with respect to $\mathcal{B}$:\\
\\
$[1-2t^2-3t^3]_\mathcal{B} =
\begin{bmatrix}
    -3 \\ -2 \\ 0 \\ 1
\end{bmatrix}$
\hfill
$[t + 2t^3]_\mathcal{B} =
\begin{bmatrix}
    2 \\ 0 \\ 1 \\ 0
\end{bmatrix}$
\hfill
$[1 + t -2t^2]_\mathcal{B} =
\begin{bmatrix}
    0 \\ -2 \\ 1 \\ 1
\end{bmatrix}$\\
\\[0.1in]
To determine if these polynomials are linearly independent, we can put the column vectors into a matrix and determine if there is a pivot
in every column:\\
\\
$
\begin{bmatrix}
    -3 & 2 & 0 \\
    -2 & 0 & -2 \\
    0 & 1 & 1 \\
    1 & 0 & 1
\end{bmatrix}
\hfill
\backsim
\hfill
\begin{bmatrix}
    -3 & 2 & 0 \\
    0 & 0 & 0 \\
    0 & 1 & 1 \\
    1 & 0 & 1
\end{bmatrix}
\hfill
\backsim
\hfill
\begin{bmatrix}
    -3 & 2 & 0 \\
    1 & 0 & 1 \\
    0 & 1 & 1 \\
    0 & 0 & 0
\end{bmatrix}
\hfill
\backsim
\hfill
\begin{bmatrix}
    -3 & 2 & 0 \\
    1 & 0 & 1 \\
    -1 & 1 & 0 \\
    0 & 0 & 0
\end{bmatrix}
\backsim\\
\\[0.07in]
\begin{bmatrix}
    1 & 0 & 0 \\
    0 & 0 & 1 \\
    0 & 1 & 0 \\
    0 & 0 & 0
\end{bmatrix}
\backsim
\begin{bmatrix}
    1 & 0 & 0 \\
    0 & 1 & 0 \\
    0 & 0 & 1 \\
    0 & 0 & 0
\end{bmatrix}$\\
\\[0.1in]
Because there is a pivot in every column, we know that the polynomials are linearly independent.

\subsection*{Problem 31}

\medium \textbf{Part a}\\
\\
We will use $\mathcal{B} = \{t^2, t, 1\}$ as a basis for $\mathbb{P}_2$ to rewrite these polynomials as column vectors:\\
\\
$[1-t^2]_\mathcal{B} =
\begin{bmatrix}
    -1 \\ 0 \\ 1
\end{bmatrix}$
\hfill
$[1-3t+5t^2]_\mathcal{B} =
\begin{bmatrix}
    5 \\ -3 \\ 1
\end{bmatrix}$\\
\\
$[-3+5t-7t^2]_\mathcal{B} =
\begin{bmatrix}
    -3 \\ 5 \\ -7
\end{bmatrix}$
\hfill
$[-4+5t-6t^2]_\mathcal{B} =
\begin{bmatrix}
    -6 \\ 5 \\ -4
\end{bmatrix}$\\
\\[0.05in]
We will now combine these column vectors into a matrix and check for a pivot in each row:\\
\\
$
\begin{bmatrix}
    -1 & 5 & -3 & -6 \\
    0 & -3 & 5 & 5 \\
    1 & 1 & -7 & 4
\end{bmatrix}
\backsim
\begin{bmatrix}
    1 & 0 & 0 & 0 \\
    0 & 0 & 1 & 0 \\
    0 & 0 & 0 & 1
\end{bmatrix}$\\
\\
Because there is a pivot in every row, we know that these column vectors span $\mathbb{R}^2$. However, since these column vectors
represent polynomials in $\mathbb{P}_2$, we also know that the polynomials span $\mathbb{P}_2$.\\
\\[0.1in]
\medium \textbf{Part b}\\
\\
We will use $\mathcal{B} = \{t^2, t, 1\}$ as a basis for $\mathbb{P}_2$ to rewrite these polynomials as column vectors:\\
\\
$[5t + t^2]_\mathcal{B} =
\begin{bmatrix}
    1 \\ 5 \\ 0
\end{bmatrix}$
\hfill
$[1 - 8t - 2t^2]_\mathcal{B} =
\begin{bmatrix}
    -2 \\ -8 \\ 1
\end{bmatrix}$\\
\\
$[-3 + 4t + 2t^2]_\mathcal{B} =
\begin{bmatrix}
    2 \\ 4 \\ -3
\end{bmatrix}$
\hfill
$[1 - t^2]_\mathcal{B} =
\begin{bmatrix}
    -1 \\ 0 \\ 1
\end{bmatrix}$

\end{document}

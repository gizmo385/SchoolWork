\documentclass{article}%
\usepackage{amsmath}
\usepackage{amsfonts}
\usepackage{amssymb}
\usepackage{graphicx}
\usepackage{tikz}
\usepackage{hyperref}%
\setcounter{MaxMatrixCols}{30}
%TCIDATA{OutputFilter=late$x2$.dll}
%TCIDATA{Version=5.00.0.2552}
%TCIDATA{CSTFile=40 LaTeX article.cst}
%TCIDATA{Created=Thursday, August 21, 2008 14:03:59}
%TCIDATA{LastRevised=Wednesday, October 01, 2014 12:46:33}
%TCIDATA{<META NAME="GraphicsSave" CONTENT="32">}
%TCIDATA{<META NAME="SaveForMode" CONTENT="1">}
%TCIDATA{<META NAME="DocumentShell" CONTENT="Standard LaTeX\Blank - Standard LaTeX Article">}
%TCIDATA{Language=American English}
\newtheorem{theorem}{Theorem}
\newtheorem{acknowledgement}[theorem]{Acknowledgement}
\newtheorem{algorithm}[theorem]{Algorithm}
\newtheorem{axiom}[theorem]{Axiom}
\newtheorem{case}[theorem]{Case}
\newtheorem{claim}[theorem]{Claim}
\newtheorem{conclusion}[theorem]{Conclusion}
\newtheorem{condition}[theorem]{Condition}
\newtheorem{conjecture}[theorem]{Conjecture}
\newtheorem{corollary}[theorem]{Corollary}
\newtheorem{criterion}[theorem]{Criterion}
\newtheorem{definition}[theorem]{Definition}
\newtheorem{example}[theorem]{Example}
\newtheorem{exercise}[theorem]{Exercise}
\newtheorem{lemma}[theorem]{Lemma}
\newtheorem{notation}[theorem]{Notation}
\newtheorem{problem}[theorem]{Problem}
\newtheorem{proposition}[theorem]{Proposition}
\newtheorem{remark}[theorem]{Remark}
\newtheorem{solution}[theorem]{Solution}
\newtheorem{summary}[theorem]{Summary}
\newenvironment{proof}[1][Proof]{\noindent\textbf{#1.} }{\ \rule{0.5em}{0.5em}}

\usepackage{fancyhdr}
\setlength\headheight{26pt}
\pagestyle{fancy}
\lhead{{\footnotesize Math 215 - Homework 7}}
\rhead{{\footnotesize Christopher Chapline}}
\begin{document}

\section{Chapter 1.9}

\subsubsection{Problem 17}

Let
$T:\mathds{R}^4 \rightarrow \mathds{R}^4$ \\

\begin{bmatrix}
    $x_1$ \\ $x_2$ \\ $x_3$ \\ $x_4$
\end{bmatrix}
\rightarrow
\begin{bmatrix}
    $x_1 + 2x_2$ \\ 0 \\ $2x_2 + x_4$ \\ $x_2 - x_4$
\end{bmatrix} \\
\\

\noindent Let's determine the image of $\vec{e_1}, \vec{e_2}, \vec{e_3},$ and $\vec{e_4}$: \\
\\
$\vec{e_1} \rightarrow$
\begin{bmatrix}
    1 \\ 0 \\ 0 \\ 0
\end{bmatrix},
$\vec{e_2} \rightarrow$
\begin{bmatrix}
    2 \\ 0 \\ 2 \\ 1
\end{bmatrix},
$\vec{e_3} \rightarrow$
\begin{bmatrix}
    0 \\ 0 \\ 0 \\ 0
\end{bmatrix},
$\vec{e_4} \rightarrow$
\begin{bmatrix}
    0 \\ 0 \\ 1 \\ -1
\end{bmatrix} \\

Given these images for $e_1, e_2, e_3,$ and $e_4$, we can redefine $T$ as such:

$T:\mathds{R}^4 \rightarrow \mathds{R}^4$ \\

\begin{bmatrix}
    $x_1$ \\ $x_2$ \\ $x_3$ \\ $x_4$
\end{bmatrix}
$\rightarrow A\vec{x}$, where $A=$
\begin{bmatrix}
    1 & 2 & 0 & 0 \\
    0 & 0 & 0 & 0 \\
    0 & 2 & 0 & 1 \\
    0 & 1 & 0 & -1
\end{bmatrix} \\
\\

\subsubsection{Problem 19}

Let
$T:\mathds{R}^3 \rightarrow \mathds{R}^2$ \\

\begin{bmatrix}
    $x_1$ \\ $x_2$ \\ $x_3$
\end{bmatrix}
\rightarrow
\begin{bmatrix}
    $x_1 - 5x_2 + 4x_3$ \\ $x_2 - 6x_3$
\end{bmatrix} \\
\\

\noindent Let's determine the image of $\vec{e_1}, \vec{e_2},$ and $\vec{e_3},$ \\
\\
$\vec{e_1} \rightarrow$
\begin{bmatrix}
    1 \\ 0
\end{bmatrix},
$\vec{e_2} \rightarrow$
\begin{bmatrix}
    -5 \\ 1
\end{bmatrix},
$\vec{e_3} \rightarrow$
\begin{bmatrix}
    4 \\ -6
\end{bmatrix} \\
\\
\noindent Given these images for $e_1, e_2,$ and $e_3,$, we can redefine $T$ as such:

$T:\mathds{R}^4 \rightarrow \mathds{R}^4$ \\

\begin{bmatrix}
    $x_1$ \\ $x_2$ \\ $x_3$
\end{bmatrix}
$\rightarrow A\vec{x}$, where $A=$
\begin{bmatrix}
    1 & -5 & 4 \\
    0 & 1 & -6
\end{bmatrix} \\
\\


\subsubsection{Problem 21}
Let
$T:\mathds{R}^2 \rightarrow \mathds{R}^2$ \\

\begin{bmatrix}
    $x_1$ \\ $x_2$
\end{bmatrix}
\rightarrow
\begin{bmatrix}
    $x_1 + x_2$ \\ $4x_1 + 5x_2$
\end{bmatrix} \\
\\

\noindent First, we must find a matrix transformation equivalent to this linear transformation: \\
\\
$\vec{e_1} \rightarrow$
\begin{bmatrix}
    1 \\ 4
\end{bmatrix},
$\vec{e_2} \rightarrow$
\begin{bmatrix}
    1 \\ 5
\end{bmatrix},
$T:\vec{x} \rightarrow A\vec{x}$ where
$A = $
\begin{bmatrix}
    1 & 1 \\
    4 & 5
\end{bmatrix} \\
\\

\noindent To find a vector $\vec{x}$ where $T\vec{x}$ = \begin{bmatrix} 3 \\ 8 \end{bmatrix}, we must find the solution to the
matrix $A$ augmented by
\begin{bmatrix}
    3 \\ 8
\end{bmatrix}: \\
\\

\noindent \begin{bmatrix}
    1 & 1 & \vline & 3 \\
    4 & 5 & \vline & 8
\end{bmatrix}
\backsim
\begin{bmatrix}
    1 & 1 & \vline & 3 \\
    0 & 1 & \vline & -4
\end{bmatrix} $R_2 -= 4R_1$
\backsim
\begin{bmatrix}
    1 & 0 & \vline & 7 \\
    0 & 1 & \vline & -4
\end{bmatrix} $R_1 -= R_2$ \\
\\

\noindent Therefore, $T\vec{x}$ = \begin{bmatrix} 3 \\ 8 \end{bmatrix} is consistent when $\vec{x} = \begin{bmatrix} 7 \\ -4 \end{bmatrix}$.

\section{Chapter 2.1}
\subsubsection{Problem 2}
Let:
$A =$
\begin{bmatrix}
    2 & 0 & -1 \\
    4 & -5 & 2
\end{bmatrix},
$B = $
\begin{bmatrix}
    7 & -5 & 1 \\
    1 & -4 & -3
\end{bmatrix},
$C = $
\begin{bmatrix}
    1 & 2 \\
    -2 & 1
\end{bmatrix},
$D = $
\begin{bmatrix}
    3 & 5 \\
    -1 & 4
\end{bmatrix}, \\
\\
and $E = $
\begin{bmatrix}
    -5 \\
    3
\end{bmatrix}\\
\\[0.3in]
$A + 3B =$
\begin{bmatrix}
    2 & 0 & -1 \\
    4 & -5 & 2
\end{bmatrix} +
\begin{bmatrix}
    21 & -15 & 3 \\
    3 & -12 & -9
\end{bmatrix} =
\begin{bmatrix}
    23 & -15 & 2 \\
    7 & -17 & -7
\end{bmatrix}\\
\\[0.2in]
$2C - 3E$ is not defined because $C$ and $E$ are different sizes.\\
\\[0.2in]
$DB =$
\begin{bmatrix}
    (21 + 5) & -15 - 20 & (3 - 15) \\
    (-7 + 4) &  5 - 16 & (-1 - 12)
\end{bmatrix} =
\begin{bmatrix}
    26 & -35 & -12 \\
    -3 & -11 & -13
\end{bmatrix}\\
\\[0.2in]
$EC$ is not defined because $E$ has 1 column and $C$ has 2 rows.

\subsubsection{Problem 4}

Let $A=$
\begin{bmatrix}
    5 & -1 & 3 \\
    -4 & 3 & -6 \\
    -3 & 1 & 2
\end{bmatrix}.\\
\\[0.2in]
$A - 5I_3$ =
\begin{bmatrix}
    5 & -1 & 3 \\
    -4 & 3 & -6 \\
    -3 & 1 & 2
\end{bmatrix} -
\begin{bmatrix}
    5 & 0 & 0 \\
    0 & 5 & 0 \\
    0 & 0 & 5
\end{bmatrix} =
\begin{bmatrix}
    0 & -1 & 3 \\
    -4 & -2 & -6 \\
    -3 & 1 & -3
\end{bmatrix}\\
\\[0.2in]
$(5I_3)A = 5(I_3A) = 5A =$
\begin{bmatrix}
    25 & -5 & 15 \\
    -20 & 15 & -30 \\
    -15 & 5 & 10
\end{bmatrix}.\\

\subsubsection{Problem 8}
The matrix $B$ would need to have 5 rows. The number of rows in the product is determined by the number of rows
in the left operand of matrix multiplication, and since $BC$ has 5 rows, so must $B$.

\subsubsection{Problem 12}
Let $A = $
\begin{bmatrix}
    3 & -6 \\
    -2 & 4
\end{bmatrix}.\\
\\[0.3in]

\subsubsection{Problem 15}
a. False. $AB$ is defined as $[A\vec{b_1} A\vec{b_2}]$.\\
\\
b. False. Each column of $AB$ is a linear combination of the columns of $A$ using weights from the corresponding
column in $B$.\\
\\
c. True. Matrix multiplication distributes over addition.\\
\\
d. True, by theorem 3.\\
\\
e. False. The transpose of a product of matrices is equal to the product of their transposes in reverse order.
\\

\subsubsection{Problem 16}
a.\\
\\
b. True. This is the definition of matrix multiplication.\\
\\
c. False. $(A^2)^T = A^TA^T$.\\
\\
d. False. $(ABC)^T = C^TB^TA^T$. This generalization is stated immediately after theorem 3.\\
\\
e. True. This is stated in theorem 3.\\
\\

\subsubsection{Problem 22}


\end{document}
